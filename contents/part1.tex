\chapter{はじめに}
  \section{背景}
    近年,豪雨や地震による土砂災害(斜面崩壊,地すべり,土石流)が多発している.1時間降水量50mm以上の大雨の平均年間発生回数は,統計初期の10年間(1976〜1985年)に比べ最近10年間(2012〜2021年)で約1.4倍に増加し,2021年では土砂災害が972件発生した\cite{背景1, 背景2}.これらの被害箇所を早急に把握することは救助活動の支援,二次災害防止及び復旧活動の支援等の観点で重要である.しかし,被害箇所は広範囲であることが多く,安全面の観点から現地での情報収集が困難である\cite{背景3}.そこで,近年災害時に安全かつ迅速に解析が可能であるリモートセンシング技術が注目されている\cite{背景4}.

    リモートセンシング技術による災害把握には主に人工衛星,有人航空機(以降,ヘリコプター),無人航空機(以降,ドローン)が用いられる.人工衛星は広範囲の把握が可能であり,画像処理において扱いが容易な直下視点の画像が入手可能である.しかし,解像度が低いため詳細な情報の入手が困難であり,天候や撮影周期によっては画像が得られないという問題がある.ヘリコプターは人工衛星に比べ災害発生直後に画像を取得でき,解像度においても優れている.しかし,金銭的コストが非常に高く,被災地域周辺に発着場が存在しなければならないという問題がある.また,保有台数が少なく災害箇所を網羅できない可能性があり,悪天候時には出動できないこともある.ドローンは解像度の高い画像を安価かつ迅速に取得が可能であるため,被害箇所の早急な把握に有効である.しかし,現状の活用事例ではドローンの操縦から取得したデータの解析までの作業が全て手動で行われており,運用に要する労力が問題となっている.また,ドローンの航空法上の規制やバッテリー等の問題により撮影範囲が狭い\cite{背景5, 背景6, 背景7}.

    災害現場ではこれらのリモートセンシング技術等によって収集した情報を元に,地図上に土砂量変位や土砂移動変位を重畳した被災状況地図が作成される.災害直後の救助活動では要救助者位置の推定・捜索,流出した建物の特定等の作業が行われる.一般的に,要救助者は建物とともに下流に流される,或いは建物付近に埋まっていることが多いため,土砂の流出起点と流出終点をベクトルで表した土砂移動変位(以降,土砂移動)を重畳した被災状況地図は上記作業において重要な指標となる.従来,土砂移動の推定には災害前後空撮画像等を用いて,地表物のランドマークや明瞭な地形変化点を基準とし目視判読を行っている.また,空中写真測量や地上測量による図化,各種計器等による計測手法も用いられているが,広範囲に適用する際に労力等のコストが大きい\cite{土砂移動解析背景1, 土砂移動解析背景2}.また,救助活動や復旧活動の準備・計画立案時において,要救助者が流された位置,土砂排除を行う際に必要な機材の数及び二次災害危険箇所等を推定することが重要である.そこで,土砂の侵食量,土砂の堆積量及び土砂量の変化箇所(以降,土砂量)を重畳した被災状況地図が上記作業における用途で用いられる.従来,土砂量の把握には土砂移動と同様に目視判読や地上測量が行われるため,労力等のコストが大きい\cite{土砂量解析背景1, 土砂量解析背景2}.



  \section{先行研究}
    \subsection*{空撮画像マッチングによる土砂移動解析}
      災害前後空撮画像を用いた画像マッチングによる変位抽出を行うCOSI-Corr(Co-registration of Optically Sensed Images and Correlation)法\cite{土砂移動解析1}では,DEM\footnote{植生や建物の標高値を含まない数値標高モデル}(Digital Elevation Model)等により位置情報を与えて補正を行う.しかし,抽出できる変位量は画素サイズにより制限される.また,空撮画像を用いるため,植生の繁茂した領域では誤抽出が発生しやすい.


    \subsection*{数値地形画像マッチングによる土砂移動解析}
      航空レーザ測量から構築されたDEMによるPIV(Particle Image Velocimetry)手法\cite{土砂移動解析2}を応用した手法や,3D-GIV(Geo-morphic Image Velocimetry)解析\cite{土砂移動解析3, 土砂移動解析4, 土砂移動解析5}を用いて土砂移動の解析を行っている事例では,航空レーザ測量は金銭的・時間的コストが高いため災害直後の救助段階での利用が困難である.また,植生やマッチングエラーの影響を受けやすく,主な適用例が地すべり等の広範囲であり,土石流等の中範囲への適用例が少ない.
      

    \subsection*{その他の手法による土砂移動解析}
      計測時の点群の座標値自体を用いて3次元のマッチングを行う手法(ICP(Iterative Closest Point)法)\cite{土砂移動解析5}では,小領域における前後点群間での配置が一致するように移動量を算出する.しかし,処理が複雑で計算時間のコストが大きい.また,解析結果が元となる計測データ同士の精度に大きく影響され,広域の変動解析には多大な労力がかかる.合成開口レーダでの衛星画像を用いたインタフェロメトリ法(InSAR)\cite{土砂移動解析6}では,広域の地表面変動を数mmオーダーで解析可能だが,中小規模の災害への適用が難しい.また,山間部では不可視領域が発生する場合や,衛星の進行方向によっては感度が低下する場合がある.


    \subsection*{航空レーザ測量での標高差分値による土砂量解析}
      \label{}
      航空レーザ測量\footnote{LiDAR(Light Detection and Ranging)或いはLP(Laster Profiler)とも呼ばれる}技術にて取得した災害前後のDEMによる標高差分値より土砂量の推定を行う手法\cite{土砂量解析1, 土砂量解析2}では,高解像なDEMを取得可能であるため高精度に土砂量を把握ができる.しかし,航空レーザ測量データによるDEMに対して単純に2時期の標高差分値を求めると,使用するデータによっては位置ずれの影響により地形変化がほとんどない地域においても大きな標高差が生じる場合がある.このような位置ずれは現地での測量を基にして補正されることが多いが,緊急時に実施することは容易ではない.


    \subsection*{空撮画像での標高差分値による土砂量解析}
      災害後の空撮画像から三次元復元により作成したDSM\footnote{植生や建物の標高値を含む数値標高モデル}(Digital Surface Model)と災害前の国土地理院の基盤地図情報数値標高モデル\cite{基盤地図情報}による標高差分値より土砂量の推定を行う手法\cite{土砂量解析3}では,航空レーザ測量での標高差分値による土砂量解析手法より低いコストにて土砂量を推定できる.しかし,建物や植生の標高値を含む数値標高モデルであるDSMを用いるため,植生の生長や建物の増設等の影響を受ける.内山らの手法\cite{土砂量解析4}では,土砂領域マスク画像\footnote{特定の領域のみを表示し,その他の領域を表示しないようにするための二値画像}を作成することによってこれらの影響を除去しているが,目視判読にて作成しているため労力を要する.また,基盤地図情報数値標高モデルの代わりに有償の災害前空撮画像から三次元復元により作成したDSMを用いているが,無償で提供されている災害前空撮画像は撮影時期が古い場合も多く,植生や建物の影響をより大きく受ける.また,処理工程及び金銭的なコストが大きい.


    \subsection*{衛星画像での堆積量推定による土砂量解析}
      災害前後衛星画像の差分値より土砂領域の崩壊面積を算出し,崩壊面積と流出土砂量の相関式を適用することによって崩壊土砂量を計算する手法\cite{土砂量解析5}では,航空レーザ測量を用いず広域の流出土砂量を把握できる.しかし,衛星画像を用いるため天候や撮影周期によっては早期に画像が得られない.また,土砂氾濫シミュレーションソフトであるFlow-R\cite{Flow-R}を用いるため,広域に対し氾濫の起点を手動で入力する必要がある.



  \section{本研究の目的}
    以上を踏まえ,本研究では土石流における災害後の空撮画像と災害前の国土地理院の基盤地図情報数値標高モデルを用い,画像特徴及び地形特徴を利用した土砂移動と土砂量の推定を行う.
    
    本研究の特徴として,コストの高い災害後航空レーザ測量データを用いない.代わりとして,災害後空撮画像から三次元復元によって作成したDSMを災害後データとして扱い,標高差分値解析を行う.災害前のデータについては,三次元復元等のコストの大きい処理が不要で,全国で整備されており\footnote{10mメッシュは全国にて整備されており,5mメッシュは主に都市域において整備されている},かつ更新頻度の高い基盤地図情報数値標高モデルを用いる.また,植生等による影響を除去するために利用する土砂領域マスク画像について,画像特徴及び地形特徴を用いることによって自動化する.



  \section{本論文の構成}
    本論文の構成を以下に示す.
    
    第1章では本研究の背景,先行研究,及び目的について述べた.

    第2章では本研究の提案手法について述べる.

    第3章では本研究の実験方法及び実験結果について述べる.

    第4章ではまとめとして本研究の結論及び今後の課題について述べる.
