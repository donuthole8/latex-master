\chapter{はじめに}
  \section{背景}
    近年,豪雨や地震による斜面崩壊,地すべり,土石流等の土砂災害が多発している.1時間降水量50mm以上の大雨の平均年間発生回数は,統計初期の10年間(1976〜1985年)に比べ最近10年間(2012〜2021年)で約1.4倍に増加し,2021年では土砂災害が972件発生した\cite{背景1, 背景2}.これらの被害箇所を早急に把握することは救助活動の支援や二次災害防止,および復旧活動の支援等の観点で重要である.しかし,被害箇所は広範囲であることが多く,安全面の観点から現地での情報収集が困難である\cite{背景3}.そこで,近年,災害時に安全かつ迅速に解析が可能であるリモートセンシング技術が注目されている\cite{背景4}.

    リモートセンシング技術による被災解析には主に人工衛星,有人航空機(以降,ヘリコプター),無人航空機(以降,ドローン)が用いられる.人工衛星は広範囲の把握が可能であり,画像処理において扱いが容易な直下視点の画像が入手可能である.しかし,解像度が低いため詳細な情報の入手が困難であり,天候や撮影周期によっては画像が得られないという問題がある.ヘリコプターは人工衛星に比べ災害発生直後に画像を取得でき,解像度においても優れている.しかし,金銭的コストが非常に高く,周囲に発着場が必要であるという問題がある.また,保有台数が少なく災害箇所を網羅できない可能性があり,悪天候時には出動できないこともある.ドローンは安価かつ迅速に解像度の高い画像の取得が可能であるため,被害箇所の早急な把握に有効である.現状の活用事例ではドローンの操縦から取得したデータの解析までの作業が全て手動で行われており,運用にかかる労力が問題となっており,撮影範囲が狭い\cite{背景5, 背景6, 背景7}.

    災害現場では収集した情報を元に,地図上に土砂量変位や土砂移動変位を重畳した被災状況地図の作成が行われる.災害直後の救助活動において要救助者位置の推定・捜索・流出家屋の特定等の作業が行われるが,要救助者が住宅とともに下流に流される,あるいは住宅付近に埋まっていることが多いため,土砂の流出起点と流出終点をベクトルで表した土砂移動結果(以降,土砂移動)を重畳した被災地図は上記作業において重要な指標となる.従来,土砂移動の推定では被災前後空撮画像等を用いて地表物のランドマークや明瞭な地形変化点を基準とした目視判読や,空中写真測量や地上測量による図化,各種計器等を用いた手法により変動量の測定を行っているが,広範囲に適用する際に労力等のコストが大きい\cite{土砂移動解析背景1, 土砂移動解析背景2}.また,救助活動や復旧活動の準備や計画立案時において,要救助者が流された位置や,土砂排除を行う際の必要な機材の数,二次災害危険箇所等を推定することが重要である.そこで,土砂の侵食量,土砂の堆積量,土砂量の変化箇所の情報(以降,土砂量)を重畳した被災地図が上記作業における用途で用いられる.従来,土砂量の把握には土砂移動同様目視判読や地上測量が行われるため,労力等のコストが大きい\cite{土砂量解析背景1, 土砂量解析背景2}.


  \section{先行研究}
    \subsection{空撮画像マッチングによる土砂移動解析}
      前後の空撮画像を用いた画像マッチングによる変位抽出を行う手法(COSI-Corr(Co-registration of Optically Sensed Images and Correlation)法)\cite{土砂移動解析1}では,DEM等により位置情報を与えて補正を行う.しかし,抽出できる変位量は画素サイズにより規制される.また,空撮画像を用いるため,植生の繁茂した領域では誤抽出が発生する.

    \subsection{数値地形画像マッチングによる土砂移動解析}
      また,航空レーザ測量から構築されたDEM(数値標高モデル)によるPIV(Particle Image Velocimetry)手法\cite{土砂移動解析2}を応用した手法や,3D-GIV(Geo-morphic Image Velocimetry)解析\cite{土砂移動解析3, 土砂移動解析4, 土砂移動解析5}を用いて土砂移動の解析を行っている事例もあるが,航空レーザ測量は時間等のコストが高いため災害直後の救助段階での利用が困難である.また,植生やマッチングエラーの影響を受けやすく,主な適用例が地すべり等の広範囲であり,土石流等の中範囲への適用例が少ない.

    \subsection{その他の手法による土砂移動解析}
      計測時の点群の座標値自体を用いて3次元のマッチングを行う手法(ICP(Iterative Closest Point)法)\cite{土砂移動解析5}では,小領域における前後の点群間でそれらの配置が一致するように移動量を算出する.しかし,処理が複雑で計算時間が多い.また,解析結果が元となる計測データ同士の精度に大きく影響され広域の変動解析には多大な労力がかかる.合成開口レーダでの衛星画像を用いたインタフェロメトリ法(InSAR)\cite{土砂移動解析6}では,広域の地表面変動を数mmオーダーで解析可能であるが,中小規模の災害への適用が難しい.また,山間部では不可視領域が発生したり,衛星の進行方向によっては感度が落ちる.

    \subsection{航空レーザ測量での標高値差分による土砂量解析}
      航空レーザ測量(LiDAR(Light Detection and Ranging)或いはLP(Laster Profiler)とも呼ばれる)技術にて取得した災害前後の数値標高モデルであるDEM(Digital Elevation Model)による標高差分値より土砂量の推定を行う手法\cite{土砂量解析1, 土砂量解析2}では,高解像であるため高精度に土砂量を把握ができる.しかし,航空レーザ測量データによるDEMに対して単純に2時期の標高値差分を求めると,使用するデータによっては,位置ずれの影響により地形変化がほとんどない地域においても大きな標高差が生じる場合がある.このような位置ずれは現地での測量を基にして補正されることが多いが,緊急時に実施することは容易ではない.また,航空レーザ測量は金銭的コスト・時間的コストが非常に高い.

    \subsection{航空画像での標高値差分による土砂量解析}
      災害後の空撮画像から三次元復元により作成した数値標高モデルであるDSM(Digital Surface Model)と災害前の国土地理院DEMによる標高差分値により土砂量の推定を行う手法\cite{土砂量解析3}では,\ref{航空レーザ測量での標高値差分による土砂量解析}項の手法より低いコストにて土砂量を推定できる.しかし,建物や植生の標高値を含む標高値モデルであるDSMを用いるため,植生の生長の影響を受ける.内山らの手法\cite{土砂量解析4}では,土砂領域のマスク画像を作成することによってこれらの影響を除去しているが,目視判読にて作成しているため労力がかかる.
    
    \subsection{衛星画像での堆積量推定による土砂量解析}
      災害前後の衛星画像の差分値より土砂領域の崩壊面積を検出し,崩壊面積と流出土砂量の相関式を適応することによって崩壊土砂の流出土砂量を計算する手法\cite{土砂量解析5}では,航空レーザ測量を用いず広域の流出土砂量を把握でる.しかし,衛星画像を用いるため天候や撮影周期によっては早期に画像が得られない.また,土砂氾濫シミュレーションソフトであるFlow−R\cite{Flow-R}を用いるため,氾濫の起点を手動で入力する必要がある.


  \section{本研究の目的}
    以上を踏まえ,本研究では土石流における災害後空撮画像と災害前の国土地理院DEM\cite{基盤地図情報}を用いて,地形情報と画像情報を併用した土砂移動と土砂量の推定を行う.本研究の特徴として,コストの高い災害後の航空レーザ測量データを用いない.また,手動作成であった土砂マスクの作成を自動で行い,植生による影響を除去する.


  \section{本論文の構成}
    本論文の構成を以下に示す.
    
    第1章では本研究の背景,関連研究,及び目的について述べた.

    第2章では本研究の提案手法について述べる.

    第3章では実験方法及び実験結果について述べる.

    第4章ではまとめとして結論及び今後の課題について述べる.
