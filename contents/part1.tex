\chapter{はじめに}
  \section{背景}
    近年,豪雨や地震による斜面崩壊,地すべり,土石流等の土砂災害が多発しており,気象庁の統計では1時間降水量50mm以上の大雨の平均年間発生回数は,統計初期の10年間(1976〜1985年)に比べ最近10年間(2011〜20XX年)で約X.X倍に増加した\cite{背景1}.XXXX〜XXXX年のX年間の間に土砂災害がXXXX件発生し,家屋や人命に大きな損害を与えた.これらの被害箇所を早急に把握することは救助活動の支援や二次災害防止,および復旧活動の支援等の観点で重要である\cite{}.しかし,被害箇所は広範囲に及ぶ上,


    そこで,災害時に安全かつ迅速に解析が可能なリモートセンシング技術が活用されており,特に迅速性が重要な救助の段階では航空機やドローンによって被災解析が行われる\cite{}.

    リモートセンシング技術による災害領域検出には主に人工衛星,有人航空機(以降,ヘリコプター),無人航空機(以降,ドローン)が用いられる.人工衛星は広範囲の把握が可能であり,画像処理において扱いが容易な直下視点の画像が入手可能である.しかし,解像度が低いため詳細な情報の入手が困難であり,天候や撮影周期によっては画像が得られないという問題がある.ヘリコプターは人工衛星に比べ災害発生直後に画像を取得でき,解像度においても優れている.しかし,金銭的コストが非常に高く,周囲に発着場が必要であるという問題がある.また,保有台数が少なく災害箇所を網羅できない可能性があり,悪天候時には出動できないこともある.ドローンは安価かつ迅速に解像度の高い画像の取得が可能であるため,被害箇所の早急な把握に有効である.しかし,現状の活用事例ではドローンの操縦から取得したデータの解析までの作業が全て手動で行われており,運用にかかる労力が問題となっている.

    災害時に安全かつ迅速に解析が可能なリモートセンシング技術が活用されており,特に迅速性が重要な救助の段階では航空機やドローンによって被災解析が行われる.また,災害直後の救助活動において要救助者位置の推定・捜索・流出家屋の特定等の作業が行われるが,要救助者が住宅とともに下流に流される,あるいは住宅付近に埋まっていることが多いため,土砂量や土砂移動の変動を示す被災地図は,上記作業において重要な指標となる.


  \section{先行研究}
    従来,土砂移動の推定では被災前後空撮画像等を用いて地表物のランドマークや明瞭な地形変化点を基準とした目視判読により変動量の測定を行っているが\cite{先行研究1},広範囲に適用する際に労力等のコストが大きい.また,航空レーザ測量から構築されたDEM(数値標高モデル)による3D-GIV(Geo-morphic Image Velocimetry)解析を用いて土砂移動の解析を行っている事例もあるが\cite{先行研究2},航空レーザ測量は時間等のコストが高いため災害直後の救助段階での利用が困難である.さらに,主な適用例が地すべり等の広範囲であり,土石流等の中範囲への適用例が少ない.

    TODO: 空中写真測量・地上測量による図化
    目視判読
    現地測量・各種計器を用いた手法にも言及する

    \subsection{多時期DEMを用いた土砂移動解析手法}
      多時期のDEMを用いて地表面の変化を抽出する手法

    \subsection{地表面変位を面的に計測する手法}


    \subsection{点群座標値を用いて3次元マッチングを行う手法}


    \subsection{ラスター型地表面モデルから画像マッチングを行う手法}


    \subsection{数値地形画像マッチング手法}


  \section{本研究の目的}
    以上を踏まえ,本研究では土石流における災害後空撮画像と災害前の国土地理院DEM\cite{使用データ1}を入力とし,地形情報と画像情報を用いることによって半自動的に土砂量と土砂移動の推定を行う.


  \section{本論文の構成}
    本論文の構成を以下に示す.
    
    第1章では本研究の背景,関連研究,及び目的について述べた.

    第2章では本研究の提案手法について述べる.
    
    第3章では実験方法及び実験結果について述べる.

    第4章ではまとめとして結論及び今後の課題について述べる.


