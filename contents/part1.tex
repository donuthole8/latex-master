\chapter{はじめに}

  \section{背景}
    近年,豪雨災害による斜面崩壊・浸水被害が多発し,これらの被害箇所を早急に把握することは救助や二次災害防止,および復旧等に有効である.災害時に安全かつ迅速に解析が可能なリモートセンシング技術が活用されており,特に迅速性が重要な救助の段階では航空機やドローンによって被災解析が行われる.また,災害直後の救助活動において要救助者位置の推定・捜索・流出家屋の特定等の作業が行われるが,要救助者が住宅とともに下流に流される,あるいは住宅付近に埋まっていることが多いため,土砂量や土砂移動の変動を示す被災地図は,上記作業において重要な指標となる.

  \section{先行研究}
    \subsection{の手法}
      従来,土砂移動の推定では被災前後空撮画像等を用いて地表物のランドマークや明瞭な地形変化点を基準とした目視判読により変動量の測定を行っているが[1],広範囲に適用する際に労力等のコストが大きい.また,航空レーザ測量から構築されたDEM(数値標高モデル)による3D-GIV(Geo-morphic Image Velocimetry)解析を用いて土砂移動の解析を行っている事例もあるが[2],航空レーザ測量は時間等のコストが高いため災害直後の救助段階での利用が困難である.さらに,主な適用例が地すべり等の広範囲であり,土石流等の中範囲への適用例が少ない.

    \subsection{の手法}

    \subsection{の手法}

  \section{本研究の目的}
    以上を踏まえ,本研究では土石流における災害後空撮画像と災害前の国土地理院DEM[3]を入力とし,地形情報と画像情報を用いることによって半自動的に土砂量と土砂移動の推定を行う.

  \section{本論文の構成}
    本論文の構成を以下に示す.
    
    第1章では本研究の背景,関連研究,及び目的について述べた.

    第2章では本研究の提案手法について述べる.
    
    第3章では実験方法及び実験結果について述べる.

    第4章ではまとめとして結論及び今後の課題について述べる.


