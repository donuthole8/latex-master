\chapter{実験}
  \section{実験環境}

  \section{入力データ}
    本研究では国立研究開発法人防災科学技術研究所様に貸与頂いた平成30年7月豪雨における広島県坂町小屋浦のドローン空撮画像[8]を使用して実験を行った.空撮画像の詳細を表3.1に,国土地理院DEMの詳細を表3.2に示す.

  \section{実験結果}
    入力画像として267枚のドローン空撮画像と国土地理院DEMを利用した(図2, 図3).
  
    まず,空撮画像に三次元復元を適用することにより生成した三次元モデルよりDSMとオルソ画像を作成し,標高値補正を行った.標高値補正済み災害後DSMを図4に,オルソ画像を図5に示す.
  
    その後,作成した土砂マスクを図6に示す.
  
    最後に,土砂量図と土砂移動図を図7,図8に示す.ここで図8の矢印は土砂位の移動方向を示す.
  
  
  \section{精度評価}
    メッシュ毎の土砂移動の方向についての精度評価を行った.

    まず,国土地理院の空中写真による被災前後の比較[9]にて目視判読にて土砂マスクの土砂部分に限定し正解データを作成した(図9). その後,2.5節での土砂移動推定結果による土砂移動方向の角度データと正解角度データを比較することにより精度評価を行った(式4).answerは正解角度データ,resultは土砂移動推定結果による角度データ,accuracyは精度を示す.

    % accuracy=1-|(answer-result)/180|-(式4)

    評価した結果,各メッシュの平均精度は0.759であった.

  \section{考察}
