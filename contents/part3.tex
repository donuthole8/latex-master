\chapter{実験}
  \section{実験環境}
    本研究における実験環境を\tref{実験環境}に示す.

    TODO: もし分かればMetashapeの実験環境を記述

    \begin{table}[t]
      \centering
      \caption{実験環境}
      \label{実験環境}
      \begin{tabular}{ll}
        \hline
        項目 & 詳細 \\
        \hline \hline
        CPU & Apple M1 \\
        メモリ & 16.00GB \\
        OS & macOS Monterey 12.4 \\
        使用言語 & Python 3.9.1 \\
        コンパイラ & インタープリタ言語により未使用 \\
        使用ライブラリ & OpenCV 4.5.5, GDAL 3.5.3, pymeanshift\cite{PyMeanShift} \\ 
        使用ソフトウェア & Metashape 1.6.4, QGIS 3.22.13 \\ \hline 
      \end{tabular}
    \end{table}


  \section{入力データ}
    \label{入力データ(実験)}
    本研究では災害後空撮画像として防災科学技術研究所提供のドローン空撮画像\cite{防災科研空撮画像}(実験1),国土地理院提供のヘリコプター空撮画像\cite{国土地理院空撮画像1,国土地理院空撮画像2}(実験2,実験3),災害前DEMとして国土地理院DEM,国土地理院の建物ポリゴンを使用して実験を行った.実験1は平成30年7月豪雨のドローン空撮画像であり,防災科学技術研究所より貸与頂いた.実験2,実験3はそれぞれ平成30年7月豪雨と熱海市伊豆山土石流災害のヘリコプター空撮画像であり,国土地理院の公開している災害情報ページより取得した.災害前DEM及び建物ポリゴンについても,国土地理院の公開しているダウンロードページより取得した.使用したデータの詳細を\tref{使用空撮画像}から\tref{使用建物ポリゴン}に示す.

    また,使用した災害後空撮画像の内3例を\fref{空撮画像1}から\fref{空撮画像3}に,災害前DEM及び建物ポリゴンを\fref{基盤地図情報(実験1)}から\fref{基盤地図情報(実験3)}に示す.なお,災害前DEM及び建物ポリゴンは後述で切り抜いた災害後DSM及びオルソ画像に合わせてトリミングを行った.

    \begin{table}[t]
      \centering
      \caption{空撮画像(実験1)}
      \label{使用空撮画像}
      \begin{tabular}{ll}
        \hline
        項目 & 詳細 \\
        \hline \hline
        災害名称 & 平成30年7月豪雨小屋浦 \\
        撮影箇所 & 広島県安芸郡坂町 \\
        撮影機体 & ドローン \\
        撮影日時 & 2018年7月10日 \\
        使用枚数 & 267枚 \\
        解像度 & 3000 $\times$ 4000 画素 \\
        提供 & 防災科学技術研究所 \\ \hline
      \end{tabular}
    \end{table}

    \begin{table}[t]
      \centering
      \caption{空撮画像(実験2)}
      \begin{tabular}{ll}
        \hline
        項目 & 詳細 \\
        \hline \hline
        災害名称 & 平成30年7月豪雨 \\
        撮影箇所 & 広島県三原市三原北部 \\
        撮影機体 & ヘリコプター \\
        撮影日時 & 2018年7月15日 \\
        使用枚数 & 39枚 \\
        解像度 & 3584 $\times$ 3932 画素 \\
        提供 & 国土地理院 \\ \hline
      \end{tabular}
    \end{table}

    \begin{table}[t]
      \centering
      \caption{空撮画像(実験3)}
      \begin{tabular}{ll}
        \hline
        項目 & 詳細 \\
        \hline \hline
        災害名称 & 熱海市伊豆山土石流災害 \\
        撮影箇所 & 静岡県熱海市伊豆山 \\
        撮影機体 & ヘリコプター \\
        撮影日時 & 2021年7月6日 \\
        使用枚数 & 39枚 \\
        解像度 & 2828 $\times$ 4328 画素 \\
        提供 & 国土地理院 \\ \hline
      \end{tabular}
    \end{table}

    \begin{table}[t]
      \centering
      \caption{災害前DEM(実験1-3)}
      \label{使用災害前DEM}
      \begin{tabular}{ll}
        \hline
        項目 & 詳細 \\
        \hline \hline
        名称 & 基盤地図情報数値標高モデル \\
        作成日時 & 
        \begin{tabular}{l}
        2016年10月01日(実験1) \\
        2016年10月01日(実験2) \\
        2019年07月01日(実験3) \\
        \end{tabular} \\
        データ形式 & 5m(標高) \\
        提供 & 国土地理院 \\ \hline
      \end{tabular}
    \end{table}

    \begin{table}[t]
      \centering
      \caption{建物ポリゴン(実験1-3)}
      \label{使用建物ポリゴン}
      \begin{tabular}{ll}
        \hline
        項目 & 詳細 \\
        \hline \hline
        名称 & 基盤地図情報基本項目建築物 \\
        作成日時 &
        \begin{tabular}{l}
        2022年07月01日(実験1) \\
        2022年10月01日(実験2) \\
        2022年10月01日(実験3) \\
        \end{tabular} \\
        縮尺 & 2500分の1 \\
        提供 & 国土地理院 \\ \hline
      \end{tabular}
    \end{table}

    \begin{figure}[t]
      \begin{minipage}[c]{0.329\hsize}
        \centering
        \includegraphics[width=4.5cm]{image/exmaple/input1.jpg}
        \subcaption{入力画像例1}
      \end{minipage}
      \begin{minipage}[c]{0.329\hsize}
        \centering
        \includegraphics[width=4.5cm]{image/exmaple/input2.jpg}
        \subcaption{入力画像例2}
      \end{minipage}
      \begin{minipage}[c]{0.329\hsize}
        \centering
        \includegraphics[width=4.5cm]{image/exmaple/input3.jpg}
        \subcaption{入力画像例3}
      \end{minipage}
      \caption{災害後空撮画像の例(実験1)}
      \label{空撮画像1}
    \end{figure}

    \begin{figure}[t]
      \begin{minipage}[c]{0.329\hsize}
        \centering
        \includegraphics[width=4.5cm]{image/exp2/input1.jpg}
        \subcaption{入力画像例1}
      \end{minipage}
      \begin{minipage}[c]{0.329\hsize}
        \centering
        \includegraphics[width=4.5cm]{image/exp2/input2.jpg}
        \subcaption{入力画像例2}
      \end{minipage}
      \begin{minipage}[c]{0.329\hsize}
        \centering
        \includegraphics[width=4.5cm]{image/exp2/input3.jpg}
        \subcaption{入力画像例3}
      \end{minipage}
      \caption{災害後空撮画像の例(実験2)}
      \label{空撮画像2}
    \end{figure}

    \begin{figure}[t]
      \begin{minipage}[c]{0.329\hsize}
        \centering
        \includegraphics[width=4.5cm]{image/exp3/input1.jpeg}
        \subcaption{入力画像例1}
      \end{minipage}
      \begin{minipage}[c]{0.329\hsize}
        \centering
        \includegraphics[width=4.5cm]{image/exp3/input2.jpeg}
        \subcaption{入力画像例2}
      \end{minipage}
      \begin{minipage}[c]{0.329\hsize}
        \centering
        \includegraphics[width=4.5cm]{image/exp3/input3.jpeg}
        \subcaption{入力画像例3}
      \end{minipage}
      \caption{災害後空撮画像の例(実験3)}
      \label{空撮画像3}
    \end{figure}

    \begin{figure}[t]
      \begin{minipage}[c]{0.45\hsize}
        \centering
        \includegraphics[width=\linewidth]{image/exmaple/dem.png}
        \subcaption{災害前DEM}
      \end{minipage}
      \begin{minipage}[c]{0.45\hsize}
        \centering
        \includegraphics[width=\linewidth]{image/exmaple/building_polygon.png}
        \subcaption{建物ポリゴン}
      \end{minipage}
      \caption{建物領域の検出}
      \label{基盤地図情報(実験1)}
    \end{figure}

    \begin{figure}[t]
      \begin{minipage}[c]{0.45\hsize}
        \centering
        \includegraphics[width=\linewidth]{image/exmaple/building_polygon.png}
        \subcaption{災害前DEM}
      \end{minipage}
      \begin{minipage}[c]{0.45\hsize}
        \centering
        \includegraphics[width=\linewidth]{image/exmaple/building.png}
        \subcaption{建物ポリゴン}
      \end{minipage}
      \caption{建物領域の検出}
      \label{基盤地図情報(実験2)}
    \end{figure}

    \begin{figure}[t]
      \begin{minipage}[c]{0.45\hsize}
        \centering
        \includegraphics[width=\linewidth]{image/exmaple/building_polygon.png}
        \subcaption{災害前DEM}
      \end{minipage}
      \begin{minipage}[c]{0.45\hsize}
        \centering
        \includegraphics[width=\linewidth]{image/exmaple/building.png}
        \subcaption{建物ポリゴン}
      \end{minipage}
      \caption{建物領域の検出}
      \label{基盤地図情報(実験3)}
    \end{figure}



  \section{実験結果}
    \subsection{三次元復元}
      \label{三次元復元}
      \ref{入力データ(実験)}節の災害後空撮画像を入力とし,Metashapeにて三次元復元処理を行った.実験1-3における写真のアラインメント,高密度クラウド構築,メッシュ構築,テクスチャ構築,Z軸指定,DEM構築,オルソモザイク構築の各処理結果を\fref{三次元復元処理結果(実験1)}から\fref{三次元復元処理結果(実験2)}に示す.
      
    \begin{figure}[t]
      \begin{tabular}{cc}
        \begin{minipage}[c]{0.45\hsize}
          \centering
          \includegraphics[width=\linewidth]{image/exmaple/alignment.jpg}
          \subcaption{写真のアラインメント結果}
        \end{minipage} &
        \begin{minipage}[c]{0.45\hsize}
          \centering
          \includegraphics[width=\linewidth]{image/exmaple/dense_cloud.jpg}
          \subcaption{高密度クラウド構築結果}
        \end{minipage} \\
        \begin{minipage}[c]{0.45\hsize}
          \centering
          \includegraphics[width=\linewidth]{image/exmaple/mesh.jpg}
          \subcaption{メッシュ構築結果}
        \end{minipage} &
        \begin{minipage}[c]{0.45\hsize}
          \centering
          \includegraphics[width=\linewidth]{image/exmaple/texture.jpg}
          \subcaption{テクスチャ構築結果}
        \end{minipage} \\
        \begin{minipage}[c]{0.45\hsize}
          \centering
          \includegraphics[width=\linewidth]{image/exmaple/view_reset.jpg}
          \subcaption{Z軸指定結果}
        \end{minipage} &
        \begin{minipage}[c]{0.45\hsize}
          \centering
          \includegraphics[width=\linewidth]{image/exmaple/dsm.jpg}
          \subcaption{DEM構築結果}
        \end{minipage} \\
        \begin{minipage}[c]{0.45\hsize}
          \centering
          \includegraphics[width=\linewidth]{image/exmaple/ortho.jpg}
          \subcaption{オルソモザイク構築結果}
        \end{minipage}
      \end{tabular}
      \caption{三次元復元結果(実験1)}
      \label{三次元復元結果(実験1)}
    \end{figure}

    \begin{figure}[t]
      \begin{tabular}{cc}
        \begin{minipage}[c]{0.45\hsize}
          \centering
          \includegraphics[width=\linewidth]{image/exp2/alignment.jpg}
          \subcaption{写真のアラインメント結果}
        \end{minipage} &
        \begin{minipage}[c]{0.45\hsize}
          \centering
          \includegraphics[width=\linewidth]{image/exp2/dense_cloud.jpg}
          \subcaption{高密度クラウド構築結果}
        \end{minipage} \\
        \begin{minipage}[c]{0.45\hsize}
          \centering
          \includegraphics[width=\linewidth]{image/exp2/mesh.jpg}
          \subcaption{メッシュ構築結果}
        \end{minipage} &
        \begin{minipage}[c]{0.45\hsize}
          \centering
          \includegraphics[width=\linewidth]{image/exp2/texture.jpg}
          \subcaption{テクスチャ構築結果}
        \end{minipage} \\
        \begin{minipage}[c]{0.45\hsize}
          \centering
          \includegraphics[width=\linewidth]{image/exp2/view_reset.jpg}
          \subcaption{Z軸指定結果}
        \end{minipage} &
        \begin{minipage}[c]{0.45\hsize}
          \centering
          \includegraphics[width=\linewidth]{image/exp2/dsm.jpg}
          \subcaption{DEM構築結果}
        \end{minipage} \\
        \begin{minipage}[c]{0.45\hsize}
          \centering
          \includegraphics[width=\linewidth]{image/exp2/ortho.jpg}
          \subcaption{オルソモザイク構築結果}
        \end{minipage}
      \end{tabular}
      \caption{三次元復元結果(実験2)}
    \end{figure}

    \begin{figure}[t]
      \begin{tabular}{cc}
        \begin{minipage}[c]{0.45\hsize}
          \centering
          \includegraphics[width=\linewidth]{image/exp3/alignment.jpg}
          \subcaption{写真のアラインメント結果}
        \end{minipage} &
        \begin{minipage}[c]{0.45\hsize}
          \centering
          \includegraphics[width=\linewidth]{image/exp3/dense_cloud.jpg}
          \subcaption{高密度クラウド構築結果}
        \end{minipage} \\
        \begin{minipage}[c]{0.45\hsize}
          \centering
          \includegraphics[width=\linewidth]{image/exp3/mesh.jpg}
          \subcaption{メッシュ構築結果}
        \end{minipage} &
        \begin{minipage}[c]{0.45\hsize}
          \centering
          \includegraphics[width=\linewidth]{image/exp3/texture.jpg}
          \subcaption{テクスチャ構築結果}
        \end{minipage} \\
        \begin{minipage}[c]{0.45\hsize}
          \centering
          \includegraphics[width=\linewidth]{image/exp3/view_reset.jpg}
          \subcaption{Z軸指定結果}
        \end{minipage} &
        \begin{minipage}[c]{0.45\hsize}
          \centering
          \includegraphics[width=\linewidth]{image/exp3/dsm.jpg}
          \subcaption{DEM構築結果}
        \end{minipage} \\
        \begin{minipage}[c]{0.45\hsize}
          \centering
          \includegraphics[width=\linewidth]{image/exp3/ortho.jpg}
          \subcaption{オルソモザイク構築結果}
        \end{minipage}
      \end{tabular}
      \caption{三次元復元結果(実験3)}
      \label{三次元復元結果(実験3)}
    \end{figure}


    \subsection{災害前DEMの前処理}
      \ref{入力データ(実験)}節の災害前DEMを入力とし,災害前DEMの前処理を行った.実験1-3におけるリサンプリング,傾斜角度,傾斜方位の各処理結果を\fref{災害前DEMの前処理(実験1)}から\fref{災害前DEMの前処理(実験3)}に示す.

      \begin{figure}[t]
        \begin{minipage}[c]{0.329\hsize}
          \centering
          \includegraphics[width=4.5cm]{image/exmaple/resampling.png}
          \subcaption{リサンプリング結果}
        \end{minipage}
        \begin{minipage}[c]{0.329\hsize}
          \centering
          \includegraphics[width=4.5cm]{image/exmaple/slope.png}
          \subcaption{傾斜角度図}
        \end{minipage}
        \begin{minipage}[c]{0.329\hsize}
          \centering
          \includegraphics[width=4.5cm]{image/exmaple/aspect.png}
          \subcaption{傾斜方位図}
        \end{minipage}
        \caption{災害前DEMの前処理結果(実験1)}
        \label{災害前DEMの前処理結果(実験1)}
      \end{figure}

      \begin{figure}[t]
        \begin{minipage}[c]{0.329\hsize}
          \centering
          \includegraphics[width=4.5cm]{image/exmaple/resampling.png}
          \subcaption{リサンプリング結果}
        \end{minipage}
        \begin{minipage}[c]{0.329\hsize}
          \centering
          \includegraphics[width=4.5cm]{image/exmaple/slope.png}
          \subcaption{傾斜角度図}
        \end{minipage}
        \begin{minipage}[c]{0.329\hsize}
          \centering
          \includegraphics[width=4.5cm]{image/exmaple/aspect.png}
          \subcaption{傾斜方位図}
        \end{minipage}
        \caption{災害前DEMの前処理結果(実験2)}
      \end{figure}

      \begin{figure}[t]
        \begin{minipage}[c]{0.329\hsize}
          \centering
          \includegraphics[width=4.5cm]{image/exmaple/resampling.png}
          \subcaption{リサンプリング結果}
        \end{minipage}
        \begin{minipage}[c]{0.329\hsize}
          \centering
          \includegraphics[width=4.5cm]{image/exmaple/slope.png}
          \subcaption{傾斜角度図}
        \end{minipage}
        \begin{minipage}[c]{0.329\hsize}
          \centering
          \includegraphics[width=4.5cm]{image/exmaple/aspect.png}
          \subcaption{傾斜方位図}
        \end{minipage}
        \caption{災害前DEMの前処理結果(実験3)}
        \label{災害前DEMの前処理結果(実験3)}
      \end{figure}


    \subsection{災害後オルソ画像の前処理}
      \label{災害後オルソ画像の前処理}
      \ref{三次元復元}項のオルソ画像と\ref{入力データ(実験)}の建物ポリゴンを入力とし,災害後オルソ画像の前処理を行った.実験1-3におけるオルソ画像の領域分割結果,カラーラベリング結果,建物領域検出の各処理結果を\fref{災害後オルソ画像の前処理(実験1)}から\fref{災害後オルソ画像の前処理結果(実験3)}に示す.

      \begin{figure}[t]
        \begin{minipage}[c]{0.329\hsize}
          \centering
          \includegraphics[width=4.5cm]{image/exmaple/mean-shift.png}
          \subcaption{領域分割結果}
        \end{minipage}
        \begin{minipage}[c]{0.329\hsize}
          \centering
          \includegraphics[width=4.5cm]{image/exmaple/labeling.png}
          \subcaption{カラーラベリング結果}
        \end{minipage}
        \begin{minipage}[c]{0.329\hsize}
          \centering
          \includegraphics[width=4.5cm]{image/exmaple/building.png}
          \subcaption{建物領域検出結果(赤)}
        \end{minipage}
        \caption{災害後オルソ画像の前処理結果(実験1)}
        \label{災害後オルソ画像の前処理結果(実験1)}
      \end{figure}

      \begin{figure}[t]
        \begin{minipage}[c]{0.329\hsize}
          \centering
          \includegraphics[width=4.5cm]{image/exmaple/dem.png}
          \subcaption{領域分割結果}
        \end{minipage}
        \begin{minipage}[c]{0.329\hsize}
          \centering
          \includegraphics[width=4.5cm]{image/exmaple/slope.png}
          \subcaption{カラーラベリング結果}
        \end{minipage}
        \begin{minipage}[c]{0.329\hsize}
          \centering
          \includegraphics[width=4.5cm]{image/exmaple/aspect.png}
          \subcaption{建物領域検出結果(赤)}
        \end{minipage}
        \caption{災害後オルソ画像の前処理結果(実験3)}      
      \end{figure}

      \begin{figure}[t]
        \begin{minipage}[c]{0.329\hsize}
          \centering
          \includegraphics[width=4.5cm]{image/exmaple/dem.png}
          \subcaption{領域分割結果}
        \end{minipage}
        \begin{minipage}[c]{0.329\hsize}
          \centering
          \includegraphics[width=4.5cm]{image/exmaple/slope.png}
          \subcaption{カラーラベリング結果}
        \end{minipage}
        \begin{minipage}[c]{0.329\hsize}
          \centering
          \includegraphics[width=4.5cm]{image/exmaple/aspect.png}
          \subcaption{建物領域検出結果(赤)}
        \end{minipage}
        \caption{災害後オルソ画像の前処理結果(実験3)}
        \label{災害後オルソ画像の前処理結果(実験3)}
      \end{figure}


    \subsection{災害後DSMの前処理}
      \ref{三次元復元}項の災害後DSMを入力とし,災害後DSMの前処理を行った.実験1-3におけるジオリファレンサ,建物領域の標高値補正,災害後DSMの正規化の各処理結果を\fref{災害後DSMの前処理(実験1)}から\fref{災害後DSMの前処理結果(実験3)}に示す.なお,実験1については位置情報が埋め込まれている画像を用いたため,ジオリファレンサを行わず自動で地図上への重畳が行われる.

      \begin{figure}[t]
        \begin{minipage}[c]{0.329\hsize}
          \centering
          \includegraphics[width=4.5cm]{image/exmaple/georeferencer.png}
          \subcaption{ジオリファレンサ結果}
        \end{minipage}
        \begin{minipage}[c]{0.329\hsize}
          \centering
          \includegraphics[width=4.5cm]{image/exmaple/removed_building.png}
          \subcaption{建物領域の標高値補正結果}
        \end{minipage}
        \begin{minipage}[c]{0.329\hsize}
          \centering
          \includegraphics[width=4.5cm]{image/exmaple/normed_dsm.png}
          \subcaption{災害後DSMの正規化結果}
        \end{minipage}
        \caption{災害後DSMの前処理結果(実験1)}
        \label{災害後DSMの前処理結果(実験1)}
      \end{figure}

      \begin{figure}[t]
        \begin{minipage}[c]{0.329\hsize}
          \centering
          \includegraphics[width=4.5cm]{image/exmaple/dem.png}
          \subcaption{ジオリファレンサ結果}
        \end{minipage}
        \begin{minipage}[c]{0.329\hsize}
          \centering
          \includegraphics[width=4.5cm]{image/exmaple/slope.png}
          \subcaption{建物領域の標高値補正結果}
        \end{minipage}
        \begin{minipage}[c]{0.329\hsize}
          \centering
          \includegraphics[width=4.5cm]{image/exmaple/aspect.png}
          \subcaption{災害後DSMの正規化結果}
        \end{minipage}
        \caption{災害後DSMの前処理結果(実験2)}
      \end{figure}

      \begin{figure}[t]
        \begin{minipage}[c]{0.329\hsize}
          \centering
          \includegraphics[width=4.5cm]{image/exmaple/dem.png}
          \subcaption{ジオリファレンサ結果}
        \end{minipage}
        \begin{minipage}[c]{0.329\hsize}
          \centering
          \includegraphics[width=4.5cm]{image/exmaple/slope.png}
          \subcaption{建物領域の標高値補正結果}
        \end{minipage}
        \begin{minipage}[c]{0.329\hsize}
          \centering
          \includegraphics[width=4.5cm]{image/exmaple/aspect.png}
          \subcaption{災害後DSMの正規化結果}
        \end{minipage}
        \caption{災害後DSMの前処理結果(実験3)}
        \label{災害後DSMの前処理結果(実験3)}
      \end{figure}


    \subsection{土砂領域マスク画像作成}
      \label{土砂領域マスク画像作成(実験)}
      \ref{災害後オルソ画像の前処理}項のオルソ画像の領域分割結果を入力とし,土砂領域マスク画像作成を行った.実験1-3におけるヒストグラム均一化,土砂候補領域検出,植生領域検出,土砂領域検出,土砂領域マスク画像作成の各処理結果を\fref{土砂領域マスク画像作成(実験1)}から\fref{土砂領域マスク画像作成(実験3)}に示す.

      \begin{figure}[t]
        \begin{tabular}{cc}
          \begin{minipage}[c]{0.45\hsize}
            \centering
            \includegraphics[width=\linewidth]{image/exmaple/clahe.png}
            \subcaption{ヒストグラム均一化結果}
          \end{minipage} &
          \begin{minipage}[c]{0.45\hsize}
            \centering
            \includegraphics[width=\linewidth]{image/exmaple/sediment_candidate.png}
            \subcaption{土砂候補領域検出結果(赤)}
          \end{minipage} \\
          \begin{minipage}[c]{0.45\hsize}
            \centering
            \includegraphics[width=\linewidth]{image/exmaple/vegetation.png}
            \subcaption{植生領域検出結果(緑)}
          \end{minipage} &
          \begin{minipage}[c]{0.45\hsize}
            \centering
            \includegraphics[width=\linewidth]{image/exmaple/sediment.png}
            \subcaption{土砂領域検出結果(赤)}
          \end{minipage} \\
          \begin{minipage}[c]{0.45\hsize}
            \centering
            \includegraphics[width=\linewidth]{image/exmaple/normed_mask.png}
            \subcaption{土砂領域マスク画像}
          \end{minipage} &
        \end{tabular}
        \caption{土砂領域マスク画像作成(実験1)}
        \label{土砂領域マスク画像作成(実験1)}
      \end{figure}


    \subsection{土砂量推定}
      \ref{災害前DEMの前処理}項,\ref{災害後DSMの前処理}項,\ref{土砂領域マスク画像作成(実験)}項の出力結果より土砂量推定を行った.土砂量推定結果を\fref{土砂量推定(実験1)}から\fref{土砂量推定(実験3)}に示す.これらにおける疑似カラーバーを\fref{}に示す.
      
      \begin{figure}[t]
        \centering
        \includegraphics[width=9cm]{image/exmaple/difference.png}
        \caption{土砂量推定結果(実験1)}
        \label{土砂量推定結果(実験1)}
      \end{figure}

      \begin{figure}[t]
        \centering
        \includegraphics[width=9cm]{image/exmaple/difference.png}
        \caption{土砂量推定結果(実験2)}
      \end{figure}

      \begin{figure}[t]
        \centering
        \includegraphics[width=9cm]{image/exmaple/difference.png}
        \caption{土砂量推定結果(実験3)}
        \label{土砂量推定結果(実験3)}
      \end{figure}


    \subsection{土砂移動推定}
      \ref{}項,\ref{}項,\ref{}項の出力結果より土砂移動推定を行った.土砂移動推定結果を\fref{土砂移動推定(実験1)}から\fref{土砂移動推定(実験3)}に示す.ここで,矢印は土砂移動の水平移動方向を示す.

      \begin{figure}[t]
        \centering
        \includegraphics[width=9cm]{image/exmaple/sediment_vector.png}
        \caption{土砂移動推定結果(実験1)}
        \label{土砂移動推定結果(実験1)}
      \end{figure}

      \begin{figure}[t]
        \centering
        \includegraphics[width=9cm]{image/exmaple/difference.png}
        \caption{土砂移動推定結果(実験2)}
      \end{figure}

      \begin{figure}[t]
        \centering
        \includegraphics[width=9cm]{image/exmaple/difference.png}
        \caption{土砂移動推定結果(実験3)}
        \label{土砂移動推定結果(実験3)}
      \end{figure}


  \section{精度評価}
    メッシュ毎の土砂移動の方向についての精度評価を行った.

    まず,国土地理院の空中写真による被災前後の比較\cite{国土地理院空撮画像1, 国土地理院空撮画像2}にて目視判読にて土砂マスクの土砂部分に限定し正解データを作成した(\fref{}). その後,\ref{土砂移動推定}節での土砂移動推定結果による土砂移動方向の角度データと正解角度データを比較することにより精度評価を行った(\Fref{精度評価}).answerは正解角度データ,resultは土砂移動推定結果による角度データ,accuracyは精度を示す.
    
    \begin{equation}
      \label{精度評価}
      accuracy = 1 - |\dfrac{answer - result} {180}|
    \end{equation}

    評価した結果,各メッシュの平均精度は0.759であった.


    
  \section{考察}
