\chapter{実験}
  \section{実験環境}
    本研究における実験環境を\tref{実験環境1}と\tref{実験環境2}に示す.Metashapeによる三次元復元の処理は\tref{実験環境1},その他の画像処理のプログラム実行及びQGISによる処理は\tref{実験環境2}に示す.

    \begin{table}[t]
      \centering
      \caption{実験環境}
      \label{実験環境1}
      \begin{tabular}{ll}
        \hline
        \textbf{項目} & \textbf{詳細} \\
        \hline \hline
        CPU & AMD $\rm{Ryzen^{TM}}$ 7 1700 \\
        GPU & GeForce GTX 1080 \\
        メモリ & 32.00GB \\
        OS & ubuntu? XX(研究室で調べて加筆します) \\
        使用ソフトウェア & Metashape 1.6.4 \\ \hline 
      \end{tabular}
    \end{table}

    \begin{table}[t]
      \centering
      \caption{実験環境}
      \label{実験環境2}
      \begin{tabular}{ll}
        \hline
        \textbf{項目} & \textbf{詳細} \\
        \hline \hline
        CPU & Apple M1 \\
        メモリ & 16.00GB \\
        OS & macOS Monterey 12.4 \\
        使用言語 & Python 3.9.1 \\
        コンパイラ & 未使用 \\
        使用ライブラリ & OpenCV 4.5.5, GDAL 3.5.3 \\ 
        使用ソフトウェア & QGIS 3.22.13 \\ \hline 
      \end{tabular}
    \end{table}



  \section{入力データ}
    \label{入力データ(実験)}
    \subsection*{災害後空撮画像}
      災害後空撮画像として防災科学技術研究所様提供のドローン空撮画像\cite{防災科研空撮画像}(実験1),国土地理院提供のヘリコプター空撮画像\cite{国土地理院空撮画像1,国土地理院空撮画像2}(実験2,実験3),災害前DEMとして国土地理院DEM,国土地理院の建物ポリゴンを使用して実験を行った.実験1は平成30年7月豪雨のドローン空撮画像であり,防災科学技術研究所より貸与頂いた.実験2,実験3はそれぞれ平成30年7月豪雨と熱海市伊豆山土石流災害のヘリコプター空撮画像であり,国土地理院の公開している災害情報ページより取得した.それぞれの災害後空撮画像の詳細を\tref{空撮画像(実験1)}から\tref{空撮画像(実験3)}に示す.また,使用した災害後空撮画像の内3例を\fref{空撮画像1}から\fref{空撮画像3}に示す.

      \begin{table}[t]
        \centering
        \caption{空撮画像(実験1)}
        \label{空撮画像(実験1)}
        \begin{tabular}{ll}
          \hline
          \textbf{項目} & \textbf{詳細} \\
          \hline \hline
          災害名称 & 平成30年7月豪雨小屋浦 \\
          撮影箇所 & 広島県安芸郡坂町 \\
          撮影機体 & ドローン \\
          撮影日時 & 2018年7月10日 \\
          使用枚数 & 267枚 \\
          解像度 & 3000 $\times$ 4000 画素 \\
          提供 & 防災科学技術研究所 \\ \hline
        \end{tabular}
      \end{table}

      \begin{table}[t]
        \centering
        \caption{空撮画像(実験2)}
        \begin{tabular}{ll}
          \hline
          \textbf{項目} & \textbf{詳細} \\
          \hline \hline
          災害名称 & 平成30年7月豪雨 \\
          撮影箇所 & 広島県三原市三原北部 \\
          撮影機体 & ヘリコプター \\
          撮影日時 & 2018年7月15日 \\
          使用枚数 & 39枚 \\
          解像度 & 3584 $\times$ 3932 画素 \\
          提供 & 国土地理院 \\ \hline
        \end{tabular}
      \end{table}

      \begin{table}[t]
        \centering
        \caption{空撮画像(実験3)}
        \label{空撮画像(実験3)}
        \begin{tabular}{ll}
          \hline
          \textbf{項目} & \textbf{詳細} \\
          \hline \hline
          災害名称 & 熱海市伊豆山土石流災害 \\
          撮影箇所 & 静岡県熱海市伊豆山 \\
          撮影機体 & ヘリコプター \\
          撮影日時 & 2021年7月6日 \\
          使用枚数 & 39枚 \\
          解像度 & 2828 $\times$ 4328 画素 \\
          提供 & 国土地理院 \\ \hline
        \end{tabular}
      \end{table}

      \begin{figure}[t]
        \begin{minipage}[c]{0.329\hsize}
          \centering
          \includegraphics[width=4.5cm]{image/exmaple/input1.jpg}
          \subcaption{入力画像例1}
        \end{minipage}
        \begin{minipage}[c]{0.329\hsize}
          \centering
          \includegraphics[width=4.5cm]{image/exmaple/input2.jpg}
          \subcaption{入力画像例2}
        \end{minipage}
        \begin{minipage}[c]{0.329\hsize}
          \centering
          \includegraphics[width=4.5cm]{image/exmaple/input3.jpg}
          \subcaption{入力画像例3}
        \end{minipage}
        \caption{災害後空撮画像の例(実験1)}
        \label{空撮画像1}
      \end{figure}

      \begin{figure}[t]
        \begin{minipage}[c]{0.329\hsize}
          \centering
          \includegraphics[width=4.5cm]{image/exp2/input1.jpg}
          \subcaption{入力画像例1}
        \end{minipage}
        \begin{minipage}[c]{0.329\hsize}
          \centering
          \includegraphics[width=4.5cm]{image/exp2/input2.jpg}
          \subcaption{入力画像例2}
        \end{minipage}
        \begin{minipage}[c]{0.329\hsize}
          \centering
          \includegraphics[width=4.5cm]{image/exp2/input3.jpg}
          \subcaption{入力画像例3}
        \end{minipage}
        \caption{災害後空撮画像の例(実験2)}
        \label{空撮画像2}
      \end{figure}

      \begin{figure}[t]
        \begin{minipage}[c]{0.329\hsize}
          \centering
          \includegraphics[width=4.5cm]{image/exp3/input1.jpeg}
          \subcaption{入力画像例1}
        \end{minipage}
        \begin{minipage}[c]{0.329\hsize}
          \centering
          \includegraphics[width=4.5cm]{image/exp3/input2.jpeg}
          \subcaption{入力画像例2}
        \end{minipage}
        \begin{minipage}[c]{0.329\hsize}
          \centering
          \includegraphics[width=4.5cm]{image/exp3/input3.jpeg}
          \subcaption{入力画像例3}
        \end{minipage}
        \caption{災害後空撮画像の例(実験3)}
        \label{空撮画像3}
      \end{figure}


    \subsection*{基盤地図情報}
      災害前DEM及び建物ポリゴンとして国土地理院の公開しているダウンロードページより取得したデータを利用した.それぞれのデータの詳細を\tref{災害前DEM(実験1-3)}及び\tref{建物ポリゴン(実験1-3)}に示す.また,災害前DEM及び建物ポリゴンを\fref{基盤地図情報(実験1)}から\fref{基盤地図情報(実験3)}に示す.なお,災害前DEM及び建物ポリゴンは後述の災害後DSM及びオルソ画像に合わせてトリミングを行った.

      \begin{table}[t]
        \centering
        \caption{災害前DEM(実験1-3)}
        \label{災害前DEM(実験1-3)}
        \begin{tabular}{ll}
          \hline
          \textbf{項目} & \textbf{詳細} \\
          \hline \hline
          名称 & 基盤地図情報数値標高モデル \\
          作成日時 & 
          \begin{tabular}{l}
          2016年10月01日(実験1) \\
          2016年10月01日(実験2) \\
          2019年07月01日(実験3) \\
          \end{tabular} \\
          データ形式 & 5m(標高) \\
          提供 & 国土地理院 \\ \hline
        \end{tabular}
      \end{table}
  
      \begin{table}[t]
        \centering
        \caption{建物ポリゴン(実験1-3)}
        \label{建物ポリゴン(実験1-3)}
        \begin{tabular}{ll}
          \hline
          \textbf{項目} & \textbf{詳細} \\
          \hline \hline
          名称 & 基盤地図情報基本項目建築物 \\
          作成日時 &
          \begin{tabular}{l}
          2022年07月01日(実験1) \\
          2022年10月01日(実験2) \\
          2022年10月01日(実験3) \\
          \end{tabular} \\
          縮尺 & 2500分の1 \\
          提供 & 国土地理院 \\ \hline
        \end{tabular}
      \end{table}
    
      \begin{figure}[t]
        \begin{minipage}[c]{0.45\hsize}
          \centering
          \includegraphics[width=\linewidth]{image/exmaple/dem.png}
          \subcaption{災害前DEM}
        \end{minipage}
        \begin{minipage}[c]{0.45\hsize}
          \centering
          \includegraphics[width=\linewidth]{image/exmaple/building_polygon.png}
          \subcaption{建物ポリゴン}
        \end{minipage}
        \caption{建物領域検出}
        \label{基盤地図情報(実験1)}
      \end{figure}

      \begin{figure}[t]
        \begin{minipage}[c]{0.45\hsize}
          \centering
          \includegraphics[width=\linewidth]{image/exmaple/building_polygon.png}
          \subcaption{災害前DEM}
        \end{minipage}
        \begin{minipage}[c]{0.45\hsize}
          \centering
          \includegraphics[width=\linewidth]{image/exmaple/building.png}
          \subcaption{建物ポリゴン}
        \end{minipage}
        \caption{建物領域検出}
        \label{基盤地図情報(実験2)}
      \end{figure}

      \begin{figure}[t]
        \begin{minipage}[c]{0.45\hsize}
          \centering
          \includegraphics[width=\linewidth]{image/exmaple/building_polygon.png}
          \subcaption{災害前DEM}
        \end{minipage}
        \begin{minipage}[c]{0.45\hsize}
          \centering
          \includegraphics[width=\linewidth]{image/exmaple/building.png}
          \subcaption{建物ポリゴン}
        \end{minipage}
        \caption{建物領域検出}
        \label{基盤地図情報(実験3)}
      \end{figure}


    \subsection*{正解データ}
      精度評価において使用する正解データを目視判読にて作成した.国土地理院の空中写真による被災前後の比較\cite{国土地理院空撮画像1, 国土地理院空撮画像2}による目視判読を行い,後述の土砂マスクの土砂部分に限定した.作成した正解データを\fref{正解データ(実験1)}から\fref{正解データ(実験3)}に示す.

      \begin{figure}[t]
        \centering
        \includegraphics[width=9cm]{image/exp1/answer.png}
        \caption{正解データ(実験1)}
        \label{正解データ(実験1)}
      \end{figure}

      \begin{figure}[t]
        \centering
        \includegraphics[width=9cm]{image/exp1/answer.png}
        \caption{正解データ(実験2)}
      \end{figure}

      \begin{figure}[t]
        \centering
        \includegraphics[width=9cm]{image/exp1/answer.png}
        \caption{正解データ(実験3)}
        \label{正解データ(実験3)}
      \end{figure}


  \section{実験結果}
    \label{実験結果}
    \subsection*{三次元復元}
      \label{三次元復元}
      \ref{入力データ(実験)}節の災害後空撮画像を入力とし,Metashapeにて三次元復元処理を行った.実験1-3における写真のアラインメント,高密度クラウド構築,メッシュ構築,テクスチャ構築,Z軸指定,DEM構築,オルソモザイク構築の各処理結果を\fref{三次元復元結果(実験1)}から\fref{三次元復元結果(実験3)}に示す.

      \begin{figure}[t]
        \begin{tabular}{cc}
          \begin{minipage}[c]{0.45\hsize}
            \centering
            \includegraphics[width=\linewidth]{image/exmaple/alignment.jpg}
            \subcaption{写真のアラインメント結果}
          \end{minipage} &
          \begin{minipage}[c]{0.45\hsize}
            \centering
            \includegraphics[width=\linewidth]{image/exmaple/dense_cloud.jpg}
            \subcaption{高密度クラウド構築結果}
          \end{minipage} \\
          \begin{minipage}[c]{0.45\hsize}
            \centering
            \includegraphics[width=\linewidth]{image/exmaple/mesh.jpg}
            \subcaption{メッシュ構築結果}
          \end{minipage} &
          \begin{minipage}[c]{0.45\hsize}
            \centering
            \includegraphics[width=\linewidth]{image/exmaple/texture.jpg}
            \subcaption{テクスチャ構築結果}
          \end{minipage} \\
          \begin{minipage}[c]{0.45\hsize}
            \centering
            \includegraphics[width=\linewidth]{image/exmaple/view_reset.jpg}
            \subcaption{Z軸指定結果}
          \end{minipage} &
          \begin{minipage}[c]{0.45\hsize}
            \centering
            \includegraphics[width=\linewidth]{image/exmaple/dsm.jpg}
            \subcaption{DEM構築結果}
          \end{minipage} \\
          \begin{minipage}[c]{0.45\hsize}
            \centering
            \includegraphics[width=\linewidth]{image/exmaple/ortho.jpg}
            \subcaption{オルソモザイク構築結果}
          \end{minipage}
        \end{tabular}
        \caption{三次元復元結果(実験1)}
        \label{三次元復元結果(実験1)}
      \end{figure}

      \begin{figure}[t]
        \begin{tabular}{cc}
          \begin{minipage}[c]{0.45\hsize}
            \centering
            \includegraphics[width=\linewidth]{image/exp2/alignment.jpg}
            \subcaption{写真のアラインメント結果}
          \end{minipage} &
          \begin{minipage}[c]{0.45\hsize}
            \centering
            \includegraphics[width=\linewidth]{image/exp2/dense_cloud.jpg}
            \subcaption{高密度クラウド構築結果}
          \end{minipage} \\
          \begin{minipage}[c]{0.45\hsize}
            \centering
            \includegraphics[width=\linewidth]{image/exp2/mesh.jpg}
            \subcaption{メッシュ構築結果}
          \end{minipage} &
          \begin{minipage}[c]{0.45\hsize}
            \centering
            \includegraphics[width=\linewidth]{image/exp2/texture.jpg}
            \subcaption{テクスチャ構築結果}
          \end{minipage} \\
          \begin{minipage}[c]{0.45\hsize}
            \centering
            \includegraphics[width=\linewidth]{image/exp2/view_reset.jpg}
            \subcaption{Z軸指定結果}
          \end{minipage} &
          \begin{minipage}[c]{0.45\hsize}
            \centering
            \includegraphics[width=\linewidth]{image/exp2/dsm.jpg}
            \subcaption{DEM構築結果}
          \end{minipage} \\
          \begin{minipage}[c]{0.45\hsize}
            \centering
            \includegraphics[width=\linewidth]{image/exp2/ortho.jpg}
            \subcaption{オルソモザイク構築結果}
          \end{minipage}
        \end{tabular}
        \caption{三次元復元結果(実験2)}
      \end{figure}

      \begin{figure}[t]
        \begin{tabular}{cc}
          \begin{minipage}[c]{0.45\hsize}
            \centering
            \includegraphics[width=\linewidth]{image/exp3/alignment.jpg}
            \subcaption{写真のアラインメント結果}
          \end{minipage} &
          \begin{minipage}[c]{0.45\hsize}
            \centering
            \includegraphics[width=\linewidth]{image/exp3/dense_cloud.jpg}
            \subcaption{高密度クラウド構築結果}
          \end{minipage} \\
          \begin{minipage}[c]{0.45\hsize}
            \centering
            \includegraphics[width=\linewidth]{image/exp3/mesh.jpg}
            \subcaption{メッシュ構築結果}
          \end{minipage} &
          \begin{minipage}[c]{0.45\hsize}
            \centering
            \includegraphics[width=\linewidth]{image/exp3/texture.jpg}
            \subcaption{テクスチャ構築結果}
          \end{minipage} \\
          \begin{minipage}[c]{0.45\hsize}
            \centering
            \includegraphics[width=\linewidth]{image/exp3/view_reset.jpg}
            \subcaption{Z軸指定結果}
          \end{minipage} &
          \begin{minipage}[c]{0.45\hsize}
            \centering
            \includegraphics[width=\linewidth]{image/exp3/dsm.jpg}
            \subcaption{DEM構築結果}
          \end{minipage} \\
          \begin{minipage}[c]{0.45\hsize}
            \centering
            \includegraphics[width=\linewidth]{image/exp3/ortho.jpg}
            \subcaption{オルソモザイク構築結果}
          \end{minipage}
        \end{tabular}
        \caption{三次元復元結果(実験3)}
        \label{三次元復元結果(実験3)}
      \end{figure}


    \subsection*{災害前DEMの前処理}
      \ref{入力データ(実験)}節の災害前DEMを入力とし,災害前DEMの前処理を行った.実験1-3におけるリサンプリング,傾斜角度,傾斜方位の各処理結果を\fref{災害前DEMの前処理結果(実験1)}から\fref{災害前DEMの前処理結果(実験3)}に示す.

      \begin{figure}[t]
        \begin{minipage}[c]{0.329\hsize}
          \centering
          \includegraphics[width=4.5cm]{image/exmaple/resampling.png}
          \subcaption{リサンプリング結果}
        \end{minipage}
        \begin{minipage}[c]{0.329\hsize}
          \centering
          \includegraphics[width=4.5cm]{image/exmaple/slope.png}
          \subcaption{傾斜角度モデル}
        \end{minipage}
        \begin{minipage}[c]{0.329\hsize}
          \centering
          \includegraphics[width=4.5cm]{image/exmaple/aspect.png}
          \subcaption{傾斜方位モデル}
        \end{minipage}
        \caption{災害前DEMの前処理結果(実験1)}
        \label{災害前DEMの前処理結果(実験1)}
      \end{figure}

      \begin{figure}[t]
        \begin{minipage}[c]{0.329\hsize}
          \centering
          \includegraphics[width=4.5cm]{image/exmaple/resampling.png}
          \subcaption{リサンプリング結果}
        \end{minipage}
        \begin{minipage}[c]{0.329\hsize}
          \centering
          \includegraphics[width=4.5cm]{image/exmaple/slope.png}
          \subcaption{傾斜角度モデル}
        \end{minipage}
        \begin{minipage}[c]{0.329\hsize}
          \centering
          \includegraphics[width=4.5cm]{image/exmaple/aspect.png}
          \subcaption{傾斜方位モデル}
        \end{minipage}
        \caption{災害前DEMの前処理結果(実験2)}
      \end{figure}

      \begin{figure}[t]
        \begin{minipage}[c]{0.329\hsize}
          \centering
          \includegraphics[width=4.5cm]{image/exmaple/resampling.png}
          \subcaption{リサンプリング結果}
        \end{minipage}
        \begin{minipage}[c]{0.329\hsize}
          \centering
          \includegraphics[width=4.5cm]{image/exmaple/slope.png}
          \subcaption{傾斜角度モデル}
        \end{minipage}
        \begin{minipage}[c]{0.329\hsize}
          \centering
          \includegraphics[width=4.5cm]{image/exmaple/aspect.png}
          \subcaption{傾斜方位モデル}
        \end{minipage}
        \caption{災害前DEMの前処理結果(実験3)}
        \label{災害前DEMの前処理結果(実験3)}
      \end{figure}


    \subsection*{災害後オルソ画像の前処理}
      \label{災害後オルソ画像の前処理}
      \ref{三次元復元}項のオルソ画像と\ref{入力データ(実験)}の建物ポリゴンを入力とし,災害後オルソ画像の前処理を行った.実験1-3におけるオルソ画像の領域分割結果,カラーラベリング結果,建物領域検出の各処理結果を\fref{災害後オルソ画像の前処理結果(実験1)}から\fref{災害後オルソ画像の前処理結果(実験3)}に示す.なお,赤色の領域が建物領域である.

      \begin{figure}[t]
        \begin{minipage}[c]{0.329\hsize}
          \centering
          \includegraphics[width=4.5cm]{image/exmaple/mean-shift.png}
          \subcaption{領域分割結果}
        \end{minipage}
        \begin{minipage}[c]{0.329\hsize}
          \centering
          \includegraphics[width=4.5cm]{image/exmaple/labeling.png}
          \subcaption{カラーラベリング結果}
        \end{minipage}
        \begin{minipage}[c]{0.329\hsize}
          \centering
          \includegraphics[width=4.5cm]{image/exmaple/building.png}
          \subcaption{建物領域検出結果}
        \end{minipage}
        \caption{災害後オルソ画像の前処理結果(実験1)}
        \label{災害後オルソ画像の前処理結果(実験1)}
      \end{figure}

      \begin{figure}[t]
        \begin{minipage}[c]{0.329\hsize}
          \centering
          \includegraphics[width=4.5cm]{image/exmaple/mean-shift.png}
          \subcaption{領域分割結果}
        \end{minipage}
        \begin{minipage}[c]{0.329\hsize}
          \centering
          \includegraphics[width=4.5cm]{image/exmaple/labeling.png}
          \subcaption{カラーラベリング結果}
        \end{minipage}
        \begin{minipage}[c]{0.329\hsize}
          \centering
          \includegraphics[width=4.5cm]{image/exmaple/building.png}
          \subcaption{建物領域検出結果}
        \end{minipage}
        \caption{災害後オルソ画像の前処理結果(実験2)}      
      \end{figure}

      \begin{figure}[t]
        \begin{minipage}[c]{0.329\hsize}
          \centering
          \includegraphics[width=4.5cm]{image/exmaple/mean-shift.png}
          \subcaption{領域分割結果}
        \end{minipage}
        \begin{minipage}[c]{0.329\hsize}
          \centering
          \includegraphics[width=4.5cm]{image/exmaple/labeling.png}
          \subcaption{カラーラベリング結果}
        \end{minipage}
        \begin{minipage}[c]{0.329\hsize}
          \centering
          \includegraphics[width=4.5cm]{image/exmaple/building.png}
          \subcaption{建物領域検出結果}
        \end{minipage}
        \caption{災害後オルソ画像の前処理結果(実験3)}
        \label{災害後オルソ画像の前処理結果(実験3)}
      \end{figure}


    \subsection*{災害後DSMの前処理}
      \ref{三次元復元}項の災害後DSMを入力とし,災害後DSMの前処理を行った.実験1-3におけるジオリファレンサ,建物領域の標高値補正,災害後DSMの正規化の各処理結果を\fref{災害後DSMの前処理結果(実験1)}から\fref{災害後DSMの前処理結果(実験3)}に示す.なお,実験1については位置情報が埋め込まれている画像を用いており自動で地図上への重畳が行われるため,ジオリファレンサ処理を行っていない.

      \begin{figure}[t]
        \begin{minipage}[c]{0.329\hsize}
          \centering
          \includegraphics[width=4.5cm]{image/exmaple/georeferencer.png}
          \subcaption{ジオリファレンサ結果}
        \end{minipage}
        \begin{minipage}[c]{0.329\hsize}
          \centering
          \includegraphics[width=4.5cm]{image/exmaple/removed_building.png}
          \subcaption{建物領域の標高値補正結果}
        \end{minipage}
        \begin{minipage}[c]{0.329\hsize}
          \centering
          \includegraphics[width=4.5cm]{image/exmaple/normed_dsm.png}
          \subcaption{災害後DSMの正規化結果}
        \end{minipage}
        \caption{災害後DSMの前処理結果(実験1)}
        \label{災害後DSMの前処理結果(実験1)}
      \end{figure}

      \begin{figure}[t]
        \begin{minipage}[c]{0.329\hsize}
          \centering
          \includegraphics[width=4.5cm]{image/exmaple/georeferencer.png}
          \subcaption{ジオリファレンサ結果}
        \end{minipage}
        \begin{minipage}[c]{0.329\hsize}
          \centering
          \includegraphics[width=4.5cm]{image/exmaple/removed_building.png}
          \subcaption{建物領域の標高値補正結果}
        \end{minipage}
        \begin{minipage}[c]{0.329\hsize}
          \centering
          \includegraphics[width=4.5cm]{image/exmaple/normed_dsm.png}
          \subcaption{災害後DSMの正規化結果}
        \end{minipage}
        \caption{災害後DSMの前処理結果(実験2)}
      \end{figure}

      \begin{figure}[t]
        \begin{minipage}[c]{0.329\hsize}
          \centering
          \includegraphics[width=4.5cm]{image/exmaple/georeferencer.png}
          \subcaption{ジオリファレンサ結果}
        \end{minipage}
        \begin{minipage}[c]{0.329\hsize}
          \centering
          \includegraphics[width=4.5cm]{image/exmaple/removed_building.png}
          \subcaption{建物領域の標高値補正結果}
        \end{minipage}
        \begin{minipage}[c]{0.329\hsize}
          \centering
          \includegraphics[width=4.5cm]{image/exmaple/normed_dsm.png}
          \subcaption{災害後DSMの正規化結果}
        \end{minipage}
        \caption{災害後DSMの前処理結果(実験3)}
        \label{災害後DSMの前処理結果(実験3)}
      \end{figure}


    \subsection*{土砂領域マスク画像作成}
      \label{土砂領域マスク画像作成(実験)}
      \ref{災害後オルソ画像の前処理}項のオルソ画像の領域分割結果を入力とし,土砂領域マスク画像作成を行った.実験1-3におけるヒストグラム均一化,土砂候補領域検出,植生領域検出,急傾斜領域検出,土砂領域検出,土砂領域マスク画像作成の各処理結果を\fref{土砂領域マスク画像作成結果(実験1)}から\fref{土砂領域マスク画像作成結果(実験3)}に示す.なお,赤色の領域が土砂候補領域及び土砂領域,緑色の領域が植生領域である.

      \begin{figure}[t]
        \begin{tabular}{cc}
          \begin{minipage}[c]{0.45\hsize}
            \centering
            \includegraphics[width=\linewidth]{image/exmaple/clahe.png}
            \subcaption{ヒストグラム均一化結果}
          \end{minipage} &
          \begin{minipage}[c]{0.45\hsize}
            \centering
            \includegraphics[width=\linewidth]{image/exmaple/sediment_candidate.png}
            \subcaption{土砂候補領域検出結果}
          \end{minipage} \\
          \begin{minipage}[c]{0.45\hsize}
            \centering
            \includegraphics[width=\linewidth]{image/exmaple/vegetation.png}
            \subcaption{植生領域検出結果}
          \end{minipage} &
          \begin{minipage}[c]{0.45\hsize}
            \centering
            \includegraphics[width=\linewidth]{image/exmaple/slope_mask.png}
            \subcaption{急傾斜領域の検出結果}
          \end{minipage} \\
          \begin{minipage}[c]{0.45\hsize}
            \centering
            \includegraphics[width=\linewidth]{image/exmaple/sediment.png}
            \subcaption{土砂領域検出結果}
          \end{minipage} &
          \begin{minipage}[c]{0.45\hsize}
            \centering
            \includegraphics[width=\linewidth]{image/exmaple/normed_mask.png}
            \subcaption{土砂領域マスク画像}
          \end{minipage} \\
        \end{tabular}
        \caption{土砂領域マスク画像作成結果(実験1)}
        \label{土砂領域マスク画像作成結果(実験1)}
      \end{figure}

      \begin{figure}[t]
        \begin{tabular}{cc}
          \begin{minipage}[c]{0.45\hsize}
            \centering
            \includegraphics[width=\linewidth]{image/exmaple/clahe.png}
            \subcaption{ヒストグラム均一化結果}
          \end{minipage} &
          \begin{minipage}[c]{0.45\hsize}
            \centering
            \includegraphics[width=\linewidth]{image/exmaple/sediment_candidate.png}
            \subcaption{土砂候補領域検出結果}
          \end{minipage} \\
          \begin{minipage}[c]{0.45\hsize}
            \centering
            \includegraphics[width=\linewidth]{image/exmaple/vegetation.png}
            \subcaption{植生領域検出結果}
          \end{minipage} &
          \begin{minipage}[c]{0.45\hsize}
            \centering
            \includegraphics[width=\linewidth]{image/exmaple/slope_mask.png}
            \subcaption{急傾斜領域の検出結果}
          \end{minipage} \\
          \begin{minipage}[c]{0.45\hsize}
            \centering
            \includegraphics[width=\linewidth]{image/exmaple/sediment.png}
            \subcaption{土砂領域検出結果}
          \end{minipage} &
          \begin{minipage}[c]{0.45\hsize}
            \centering
            \includegraphics[width=\linewidth]{image/exmaple/normed_mask.png}
            \subcaption{土砂領域マスク画像}
          \end{minipage} \\
        \end{tabular}
        \caption{土砂領域マスク画像作成結果(実験2)}
      \end{figure}

      \begin{figure}[t]
        \begin{tabular}{cc}
          \begin{minipage}[c]{0.45\hsize}
            \centering
            \includegraphics[width=\linewidth]{image/exmaple/clahe.png}
            \subcaption{ヒストグラム均一化結果}
          \end{minipage} &
          \begin{minipage}[c]{0.45\hsize}
            \centering
            \includegraphics[width=\linewidth]{image/exmaple/sediment_candidate.png}
            \subcaption{土砂候補領域検出結果}
          \end{minipage} \\
          \begin{minipage}[c]{0.45\hsize}
            \centering
            \includegraphics[width=\linewidth]{image/exmaple/vegetation.png}
            \subcaption{植生領域検出結果}
          \end{minipage} &
          \begin{minipage}[c]{0.45\hsize}
            \centering
            \includegraphics[width=\linewidth]{image/exmaple/slope_mask.png}
            \subcaption{急傾斜領域の検出結果}
          \end{minipage} \\
          \begin{minipage}[c]{0.45\hsize}
            \centering
            \includegraphics[width=\linewidth]{image/exmaple/sediment.png}
            \subcaption{土砂領域検出結果}
          \end{minipage} &
          \begin{minipage}[c]{0.45\hsize}
            \centering
            \includegraphics[width=\linewidth]{image/exmaple/normed_mask.png}
            \subcaption{土砂領域マスク画像}
          \end{minipage} \\
        \end{tabular}
        \caption{土砂領域マスク画像作成結果(実験3)}
        \label{土砂領域マスク画像作成結果(実験3)}
      \end{figure}


    \subsection*{土砂量推定}
      \ref{災害前DEMの前処理}項,\ref{災害後DSMの前処理}項,\ref{土砂領域マスク画像作成(実験)}項の出力結果より土砂量推定を行った.土砂量推定結果を\fref{土砂量推定結果(実験1)}から\fref{土砂量推定結果(実験3)}に示す.これらにおける疑似カラーバーを\fref{}に示す.
      
      \begin{figure}[t]
        \centering
        \includegraphics[width=9cm]{image/exmaple/difference.png}
        \caption{土砂量推定結果(実験1)}
        \label{土砂量推定結果(実験1)}
      \end{figure}

      \begin{figure}[t]
        \centering
        \includegraphics[width=9cm]{image/exmaple/difference.png}
        \caption{土砂量推定結果(実験2)}
      \end{figure}

      \begin{figure}[t]
        \centering
        \includegraphics[width=9cm]{image/exmaple/difference.png}
        \caption{土砂量推定結果(実験3)}
        \label{土砂量推定結果(実験3)}
      \end{figure}


    \subsection*{土砂移動推定}
      \ref{傾斜角度・傾斜方位の算出}項,\ref{災害後DSMの前処理}節,\ref{土砂領域マスク画像作成}節の出力結果より土砂移動推定を行った.土砂移動推定結果を\fref{土砂移動推定結果(実験1)}から\fref{土砂移動推定結果(実験3)}及び\fref{土砂移動推定メッシュ数}に示す.

      \begin{figure}[t]
        \centering
        \includegraphics[width=9cm]{image/exmaple/sediment_vector.png}
        \caption{土砂移動推定結果(実験1)}
        \label{土砂移動推定結果(実験1)}
      \end{figure}

      \begin{figure}[t]
        \centering
        \includegraphics[width=9cm]{image/exmaple/sediment_vector.png}
        \caption{土砂移動推定結果(実験2)}
      \end{figure}

      \begin{figure}[t]
        \centering
        \includegraphics[width=9cm]{image/exmaple/sediment_vector.png}
        \caption{土砂移動推定結果(実験3)}
        \label{土砂移動推定結果(実験3)}
      \end{figure}

      \begin{table}[b]
        \centering
        \caption{土砂移動推定メッシュ数}
        \label{土砂移動推定メッシュ数}
        \begin{tabular}{ccc}
          \hline
          \textbf{実験} & \textbf{メッシュサイズ(縦,横)} & \textbf{土砂マスク適用後メッシュ数} \\
          \hline  \hline
          実験1 & (13, 12) & 57 \\
          実験2 & (XX, XX) & XX \\
          実験3 & (XX, XX) & XX \\ \hline
        \end{tabular}
      \end{table}


  \section{精度評価}
    メッシュ毎の土砂移動の方向についての精度評価を行った.まず,\ref{入力データ(実験)}節に示す正解データを作成した.\ref{土砂移動推定}節での土砂移動推定結果による土砂移動方向の角度データと正解角度データを比較することにより,\Fref{精度評価}に示す精度評価を行った.$answer$は正解角度データ,$result$は土砂移動推定結果による角度データ,$accuracy$は精度を示す.代表的な10メッシュ及び合計平均精度を記載した精度評価結果を\Fref{精度評価結果(実験1)}から\Fref{精度評価結果(実験3)}に示す.なお,メッシュ番号の$(x,y)$は画像左上部を起点とし右下部であるほど値が大きくなるとする.

    \begin{equation}
      \label{精度評価}
      accuracy = 1 - |\dfrac{answer - result} {180}|
    \end{equation}

    \begin{table}[b]
      \centering
      \caption{精度評価結果(実験1)}
      \label{精度評価結果(実験1)}
      \begin{tabular}{cccc}
        \hline
        \textbf{メッシュ番号(x,y)} & \textbf{結果角度[$^{\circ}$]} & \textbf{正解角度[$^{\circ}$]} & \textbf{精度} \\
        \hline  \hline
        1  (0,0)  & 270 & 170 & 0.443 \\
        6  (2,0)  & 280 & 160 & 0.333 \\
        12 (4,7)  & 243 & 190 & 0.704 \\
        18 (5,8)  & 257 & 250 & 0.958 \\
        24 (6,5)  & 223 & 225 & 0.991 \\
        30 (7,2)  & 242 & 260 & 0.903 \\
        36 (8,0)  & 209 & 265 & 0.693 \\
        42 (8,6)  & 265 & 250 & 0.913 \\
        48 (9,5)  & 255 & 290 & 0.811 \\
        54 (10,4) & 261 & 320 & 0.673 \\
        合計平均 & - & - & 0.730 \\
        \hline
      \end{tabular}
    \end{table}

    \begin{table}[b]
      \centering
      \caption{精度評価結果(実験2)}
      \begin{tabular}{cccc}
        \hline
        \textbf{メッシュ番号(x,y)} & \textbf{結果角度[$^{\circ}$]} & \textbf{正解角度[$^{\circ}$]} & \textbf{精度} \\
        \hline  \hline
        1  (0,0)  & XXX & XXX & XXX \\
        合計平均 & - & - & 0.730 \\
        \hline
      \end{tabular}
    \end{table}

    \begin{table}[b]
      \centering
      \caption{精度評価結果(実験3)}
      \label{精度評価結果(実験3)}
      \begin{tabular}{cccc}
        \hline
        \textbf{メッシュ番号(x,y)} & \textbf{結果角度[$^{\circ}$]} & \textbf{正解角度[$^{\circ}$]} & \textbf{精度} \\
        \hline  \hline
        1  (0,0)  & XXX & XXX & XXX \\
        合計平均 & - & - & 0.730 \\
        \hline
      \end{tabular}
    \end{table}



  \section{考察}
    \ref{実験結果}節の土砂移動推定の精度について,本手法の問題について考察を行った.


    \subsection{土砂移動推定の精度について}
      \subsubsection*{三次元復元における解像度}
        本手法の三次元復元における写真のアラインメント,高密度クラウド構築においてはMetashapeの設定値をそれぞれ「高」に設定していたが,「最高」の設定値が存在する.それぞれ「最高」の設定値ではオリジナル解像度で処理を行うが,「高」の設定値では4分の1の解像度で処理を行う.本研究での\tref{実験環境1}に示す実験環境にてそれぞれ「最高」の設定値にて高密度クラウド構築を実行した際に途中で処理が落ちてしまう場合があったため,それぞれ「高」にて処理を行った.よって,実験環境のマシンスペックを上げることによって,より解像度の高い画像を出力でき精度を向上させることができると考える.


      \subsubsection*{建物領域検出及び建物領域の標高値補正}
        建物領域検出及び建物領域の標高値補正において,\fref{建物領域低精度例}に示すように同一の建物の屋根を別々の領域として検出してしまっており,地表面の標高値として補正できていない例がある.ここで補正した災害後DSMは土砂量推定及び土砂移動推定の制度に影響するため,この処理を改善することによって土砂移動の精度が向上する.
        
        まず,建物領域検出手法の問題について災害前の建物ポリゴンを使用していること,領域分割画像において同一の建物の屋根を別領域として検出していること,使用している指標が円形度のみであることが挙げられる.土砂災害後では建物の浸水や一部損壊等の水平位置が不変であるケースが多く,建物が流出するケースは少ないため災害前の建物ポリゴンを利用していたが,流出した家屋の検出には不適である.よって,災害後DSMや災害後オルソ画像のみから検出できることが望ましい.領域分割画像について,太陽光の光源の角度や屋根の傾斜によって同一の建物の屋根であっても色相や輝度が異なるため,それぞれ別領域として分割されている.よって,輝度値や色相補正等の前処理を適用することで精度改善されると考える.円形度について,領域分割の際に形状が変形し複雑になってしまうケースや元々複雑な形状の建物である場合に検出が難しい.よって,円形度以外のテクスチャ解析や標高値地形解析等の指標も併用した閾値処理によって精度向上が期待できる.

        次に,建物領域の標高値補正の問題について建物領域を地表面領域として誤検出し補正を行っていること,地表面領域の全画素における平均標高値が適切でない可能性があることが挙げられる.建物領域を誤検出しているケースについては,先述の建物領域検出手法の改善により精度向上が期待できる.平均標高値について,単純な平均値を取るのではなく,加重を加味した平均値や地形解析による代表的な標高値の取得が有効と考えられる.

        \begin{figure}[t]
          \begin{minipage}[c]{0.45\hsize}
            \centering
            \includegraphics[width=4.5cm]{image/consideration/building.png}
            \subcaption{建物領域検出結果}
          \end{minipage}
          \begin{minipage}[c]{0.45\hsize}
            \centering
            \includegraphics[width=9cm]{image/consideration/normed_dsm.png}
            \subcaption{建物領域の標高値補正結果}
          \end{minipage}
          \caption{同一の建物の屋根を別々の領域として検出している例}
          \label{建物領域低精度例}
        \end{figure}      


      \subsubsection*{土砂マスク画像作成}
        土砂領域を検出する際に,\fref{影領域}及び\fref{道路領域}に示すように,土砂領域中の影領域が未検出であり,無被害の道路領域を誤検出しているという問題がある.

        影領域を検出するための手法として,影領域マスク画像の作成,影画素検出手法の利用が挙げられる.影領域は輝度値が小さいという特徴を持つため,\ref{L*a*b*表色系への変換}項で示した輝度(L*)で判別することが考えられる.ただし,植生領域や道路領域,特に植生にかかる影領域も同様の特徴を持つためこれらを土砂中の影領域として検出した場合,誤検出の原因となることも考えられる.また,太陽の位置及び方向や地形特徴を利用したModel-based methodsや,影の色相及び境界から検出を行うImage-based methods等を用いることが考えれる.なお,本研究で利用するデータはジオリファレンサ等により地理的情報を持つため,親和性が高い\cite{影領域検出}.

        また,道路領域を検出する手法として,道路データの利用,テクスチャ解析が挙げられる.道路データは建物ポリゴン同様に国土地理院が提供しており,道路データの割合が多く,かつ,L*a*b*表色系のL*値の低い領域を検出することが考えられる.また,テクスチャ解析の指標である領域内の画素の不均一性を示す指標である異質度(dissimilarity)を用いることによって,画像表面における均一性が高い道路領域を検出事が考えれる.

        \begin{figure}[t]
          \begin{minipage}[c]{0.45\hsize}
            \centering
            \includegraphics[width=\linewidth]{image/consideration/shadow.png}
            \subcaption{土砂中の影領域}
          \end{minipage}
          \begin{minipage}[c]{0.45\hsize}
            \centering
            \includegraphics[width=\linewidth]{image/consideration/sediment_shadow.png}
            \subcaption{土砂領域検出結果}
          \end{minipage}
          \caption{土砂中の影領域の未検出例}
          \label{影領域}
        \end{figure}
        
        \begin{figure}[t]
          \begin{minipage}[c]{0.45\hsize}
            \centering
            \includegraphics[width=\linewidth]{image/consideration/road.png}
            \subcaption{無被害の道路領域}
          \end{minipage}
          \begin{minipage}[c]{0.45\hsize}
            \centering
            \includegraphics[width=\linewidth]{image/consideration/sediment_road.png}
            \subcaption{土砂領域検出結果}
          \end{minipage}
          \caption{無被害の道路領域の誤検出例}
          \label{道路領域}
        \end{figure}


      \subsubsection*{土砂量推定}
        本手法の土砂量推定において,実験1の例ではなだらかな住宅地が多い実験例のため堆積領域の方が侵食領域より広域に分布するが,土砂領域の大部分が侵食領域となっている.考えられる原因として前述の建物領域の標高値補正の精度が低いことが挙げられるため,提案した改善手法によって精度向上が期待できる.また,災害後DSMの正規化において,災害前DEMの最小値と最大値で災害後DSMを正規化していることが原因である可能性がある.この影響により,災害後DSMの最大標高における土砂侵食領域の標高値,最小標高値における土砂堆積領域の標高値が考慮されていないと考えられる.Miuraの手法\cite{土砂量解析5}では約650箇所の土砂災害の統計解析により土石流における平均侵食深さは0.78mに近似できるとされているため,これらを加味した正規化手法を検討する.また,未補正の災害後DSM(\fref{災害後DSM(実験1)})では,右上の土砂上流部よりも道路領域の標高値の方が高いため,最大標高値がこの道路領域を基準として正規化が行われた.よって,前述の道路領域の誤検出除去によって改善が可能である.
        
        \begin{figure}[t]
          \centering
          \includegraphics[width=9cm]{image/consideration/dsm.png}
          \caption{災害後DSM(実験1)}
          \label{災害後DSM(実験1)}
        \end{figure}


      \subsubsection*{土砂移動推定}
        本手法の土砂移動推定において,前述した手法の改善を行うこと精度の向上が期待される.また,主に建物領域を含むメッシュにおいて精度が低いため建物領域の標高値補正の改善が最重要であると考える.
