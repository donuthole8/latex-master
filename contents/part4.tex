\chapter{まとめ}
  \section{結論}
    本研究では,土石流における災害後空撮画像と災害前DEM及び建物ポリゴンを用い,地形特徴と画像特徴による土砂量及び土砂移動の推定を行う手法について提案した.また,実験にてメッシュベースでの土砂移動方向についての精度評価を行った.

    提案手法では金銭的・時間的なコストの高い航空レーザ測量DEMや労力のかかる土砂マスク画像を用いないため,災害発生時の初期段階での土砂量及び土砂移動の把握が可能になると考える.また,既存手法では大部分が広域の地すべりへの適用例であった土砂移動推定について,中範囲である土石流の災害例に適用した.土砂移動の精度について,有用な精度は未知であるが,7割を超える精度にて推定を行うことができた.


  \section{今後の課題}
    以下に今後の課題を示す.


    \subsection*{精度評価}
      土砂量推定結果の精度評価の実施,土砂移動の三次元移動距離の精度評価,実際の災害対応の際に有用となる目標精度の設定が課題として挙げられる.土砂量に関しては,航空レーザ測量等の精度の高い測量結果が存在する地域にて本手法を適用し,結果比較することによって精度評価を行う.土砂移動について,現時点では目視判読による水平距離及び垂直距離の正解データの作成が難しいため,方位のみで精度評価を行っている.また,土砂移動の方位について目視判読によって正解データを作成しているため,正確性に欠ける.よって,現地測量等による土砂移動図等の存在する被災箇所で提案手法を適用し,測量結果と比較することで方位及び水平移動距離の精度評価を行う.また,垂直距離について,航空レーザ測量等の精度の高い計測結果が存在する地域にて本手法を適用する.その後推定した水平移動距離の始点と終点において,航空レーザ測量と国土地理院DEMの標高差分値を取ることで精度評価が行えると考える.また,土砂量及び土砂移動について,実際での利活用で有用となる精度目標が未知であるので調査を行う.


    \subsection*{各手法における精度向上}
      \ref{考察}節で述べたように,各手法において改善の余地が多々あるため,これらの精度向上を目指す.また,土砂領域検出,建物領域検出については目視判読による正解データの作成が可能であるため,精度評価を実施し評価及び改善を行う..


    \subsection*{Metashapeの導入}
      提案手法で使用したMetashapeは最低でも\tref{Metashapeの動作システム要件}に示すような環境が必要である.特に,GPUを搭載する必要があるため,通常のコンピュータに比べ高いマシンスペックが必要である.また,入力画像枚数を増やし,各処理の精度を高くするためにはさらにマシンスペックを高める必要があるため,金銭的コストがかかる.また,提案手法で使用したMetashapeはProfessional版でありライセンス料もかかるため,予算の少ない機関での運用が難しい可能性がある.
      
      \begin{table}[tbp]
        \centering
        \caption{Metashapeの動作システム要件}
        \label{Metashapeの動作システム要件}
        \begin{tabular}{ll}
          \hline
          \textbf{項目} & \textbf{詳細} \\
          \hline \hline
          CPU & 4-12コア 2.00GHz以上 \\
          GPU & 1024以上のCUDAコア \\
          メモリ & 32.00GB \\
          RAM & 16-32.00GB \\ \hline
        \end{tabular}
      \end{table}
