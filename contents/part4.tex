\chapter{まとめ}
  \section{結論}
    本研究では,土石流における災害後空撮画像と災害前DEM及び建物ポリゴンより,半自動的に土砂量と土砂移動の推定を行う手法について提案した.また,実験にてメッシュベースでの土砂移動方向についての精度評価を行った.



  \section{今後の課題}
    今後の課題として,全体的な精度の向上と3次元土砂移動距離の精度評価が挙げられる.


    \subsection*{1つ目}
      建物領域抽出の手法に関し,エッジ抽出,国土地理院の建物輪郭データ等を併用することによって建物領域抽出の精度向上を目指す.また,土砂移動・土砂量推定において災害後DSMの標高値補正手法の改善や水平位置補正手法\cite{土砂量解析1}等の取り入れによって土砂量の推定を高精度に行い,土砂移動をより正確に推定することを目指す.


    \subsection*{2つ目}
      精度評価に関し,現時点では土砂移動の方向角度のみで精度評価を行っているが,今後は土砂移動の水平距離と垂直距離の精度評価を行う.また,実際の救助活動等で有用となる目標精度を設定しシステム構築を行う.

      
    \subsection*{3つ目}
      建物領域の下に存在する土砂

      土砂量の精度,土砂マスク作成の精度,前後データの解像度差異,


      DEM・建物データが整備されていることが前提
        建物:DSMで検出,機械学習,流出家屋
        建物領域で災害前のデータを使っている
        建物検出の精度が低いため機械学習を導入したい
        道路も!!

      SfMで時間がかかること
        マシンスペックがある程度必要であること
        予算の少ない消防での利用が困難


      精度評価(建物,土砂領域,土砂量)
        土砂移動:水平距離,水平距離

      DEMの解像度低い,欠損部分もある,航空画像の方が良いかも
      影
      超解像

      災害後DSMの正規化処理では,
      - 土砂侵食,土砂堆積が考慮されていない
      - 三浦の手法を参考に,平均堆積厚分を考慮しちゃう



