\chapter{手法}
  \section{概要}
    提案手法概要図を\fref{提案手法概要図}に示す.
    
    入力データとして,災害後の空撮画像複数枚,国土地理院の基盤地図情報数値標高モデル(以降,DEM or 国土地理院DEM),基盤地図情報基本項目,建築物\cite{基盤地図情報}(以降,建物ポリゴン)を用いる.

    まず,災害後空撮画像の三次元復元を行うことによってDSM(数値標高モデル)とオルソ画像\footnote{擬似的に正射変換された直下視点画像}の作成を行う.次に,前処理によって,災害前DEMと災害後DSMを補正し,災害後オルソ画像より土砂マスク画像を作成する.最終的にこれらの画像情報と数値地形情報より統合解析を行い,標高値差分解析により土砂量図を,地形解析によって土砂移動図を出力する.

    \begin{figure}[t]
      \centering
      \includegraphics[width=12cm]{image/method/processing-flow.png}
      \caption{提案手法概要図}
      \label{提案手法概要図}
    \end{figure}


  \section{入力データ}
    \label{入力データ}
    入力データとして,複数枚の災害後空撮画像,DEM,建物ポリゴンを用いる.DEMは大まかに撮影範囲が含まれると想定される範囲を取得し利用する.

    本研究で用いる災害後空撮画像は以下を満たしていることを前提とする.\fref{空撮画像例}に災害後空撮画像の例を示す.

    \begin{itemize}
      \setlength{\itemsep}{-5pt}
      \item 撮影機体がヘリコプター,或いはドローンである
      \item 撮影角度が直下視点に近い
      \item 空撮画像中に後述処理での三次元復元を行える程度の特徴量が存在する
      \item 後述処理での三次元復元を行える程度の枚数である
      \item 植生の色相が緑色に近い
      \item 該当箇所の災害前DEMと建物ポリゴンが整備されている
      \item 夜間画像ではなく明瞭に画像内の地物が確認できる
    \end{itemize}

    \begin{figure}[t]
      \begin{minipage}[c]{0.3\hsize}
        \centering
        \includegraphics[width=4.25cm]{image/exmaple/input1.jpg}
        \subcaption{入力画像例1}
        \label{入力画像例1}
      \end{minipage}
      \begin{minipage}[c]{0.3\hsize}
        \centering
        \includegraphics[width=4.25cm]{image/exmaple/input2.jpg}
        \subcaption{入力画像例2}
        \label{入力画像例2}
      \end{minipage}
      \begin{minipage}[c]{0.3\hsize}
        \centering
        \includegraphics[width=4.25cm]{image/exmaple/input3.jpg}
        \subcaption{入力画像例3}
        \label{入力画像例3}
      \end{minipage}
      \caption{災害後空撮画像の例}
      \label{空撮画像例}
    \end{figure}

    また,災害前のDEMについては5mメッシュ,10mメッシュが存在するが解像度の高い5mメッシュであることが望ましい.


  \section{三次元復元}
    まず,災害後空撮画像の三次元復元を行うことによってオルソ画像とDSMの作成を行う.三次元復元とは複数枚画像を用いて被写体の形状や距離等の3次元情報を復元する処理である.オルソ画像は,三次元復元によって得られた三次元モデルを真上から投影した状態の各画素を取得することによって得る.DSMは三次元モデルを真上から投影した状態の数値標高モデルの各標高値を取得することによって得る.本手法では復元精度の高いAgisoft社のSfM\footnote{ある対象を撮影した複数枚画像より対象の形状を復元する技術}(Structure from Motion)ソフトウェアであるMetashape\cite{Metashape}を利用する.

    \subsection{写真のアラインメント}
      \label{写真のアラインメント}
      入力画像に対して特徴点の検出を行い,特徴マッチングを行うことで疎な点群の復元を行う.植生や水域等の特徴の少ない領域が原因でマッチングに失敗する場合があるため,失敗した画像を除去し再度処理を行う.また,精度を指定することが可能であり,精度を下げると入力画像の解像度を下げることによって処理時間を短縮する.提案手法では「高」を指定し,その他の項目では既定値にて処理を行う.本処理の設定値を\fref{写真のアラインメント設定値}に,\ref{入力データ}節を用いた時の処理結果を\fref{写真のアラインメント結果}に示す.

      \begin{figure}[t]
        \begin{minipage}[c]{0.45\hsize}
          \centering
          \includegraphics[width=\linewidth]{image/setting/alignment.jpg}
          \subcaption{設定値}
          \label{写真のアラインメント設定値}
        \end{minipage}
        \begin{minipage}[c]{0.45\hsize}
          \centering
          \includegraphics[width=\linewidth]{image/exmaple/alignment.jpg}
          \subcaption{処理結果}
          \label{写真のアラインメント結果}
        \end{minipage}
        \caption{写真のアラインメント}
      \end{figure}

    \subsection{高密度クラウド構築}
      \label{高密度クラウド構築}
      \ref{写真のアラインメント}項によって得られた点群と入力画像より法線を算出し,密な点群を生成する.精度を指定することが可能であり,後述の\ref{DEM構築}項の精度に影響する.また,深度フィルタによって外れ値の除去を行うことができる.提案手法では「高」に設定し,その他の項目では既定値にて処理を行う.本処理の設定値を\fref{高密度クラウド構築設定値}に,\ref{写真のアラインメント}項の出力結果を用いた時の処理結果を\fref{高密度クラウド構築結果}に示す.

      \begin{figure}[t]
        \begin{minipage}[c]{0.45\hsize}
          \centering
          \includegraphics[width=\linewidth]{image/setting/dense_cloud.jpg}
          \subcaption{設定値}
          \label{高密度クラウド構築設定値}
        \end{minipage}
        \begin{minipage}[c]{0.45\hsize}
          \centering
          \includegraphics[width=\linewidth]{image/exmaple/dense_cloud.jpg}
          \subcaption{処理結果}
          \label{高密度クラウド構築結果}
        \end{minipage}
        \caption{高密度クラウド構築}
      \end{figure}

    \subsection{メッシュ構築}
      \label{メッシュ構築}
      \ref{高密度クラウド構築}項によって得られた点群からポリゴンメッシュの構築を行う.三次元点群から三次元モデルを復元することによって,後述の\ref{Z軸指定}項の処理が用意になる.本処理の設定値を\fref{メッシュ構築設定値}に,\ref{高密度クラウド構築}項の出力結果を用いた時の処理結果を\fref{メッシュ構築結果}に示す.

      \begin{figure}[t]
        \begin{minipage}[c]{0.45\hsize}
          \centering
          \includegraphics[width=\linewidth]{image/setting/mesh.jpg}
          \subcaption{設定値}
          \label{メッシュ構築設定値}
        \end{minipage}
        \begin{minipage}[c]{0.45\hsize}
          \centering
          \includegraphics[width=\linewidth]{image/exmaple/mesh.jpg}
          \subcaption{処理結果}
          \label{メッシュ構築結果}
        \end{minipage}
        \caption{メッシュ構築}
      \end{figure}

    \subsection{テクスチャ構築}
      \label{テクスチャ構築}
      \ref{メッシュ構築}項によって生成元の点群の色が割り当てられるが,頂点の色のみが対象であるため解像度が低い.そのため,本処理によって入力画像のテクスチャを生成したメッシュに貼り付ける.本処理の設定値を\fref{テクスチャ構築設定値}に,\ref{メッシュ構築}項の出力結果を用いた時の処理結果を\fref{テクスチャ構築結果}に示す.

      \begin{figure}[t]
        \begin{minipage}[c]{0.45\hsize}
          \centering
          \includegraphics[width=\linewidth]{image/setting/texture.jpg}
          \subcaption{設定値}
          \label{テクスチャ構築設定値}
        \end{minipage}
        \begin{minipage}[c]{0.45\hsize}
          \centering
          \includegraphics[width=\linewidth]{image/exmaple/texture.jpg}
          \subcaption{処理結果}
          \label{テクスチャ構築結果}
        \end{minipage}
        \caption{テクスチャ構築}
      \end{figure}

    \subsection{Z軸指定}
      \label{Z軸指定}
      \ref{テクスチャ構築}項によって得られた三次元モデルが直下視となるよう,Metashapeにてビューを指定する.本研究では直下視点画像を入力としてるため,「ビューをリセット」を実行することで直下視点のビューに指定できる.これにより,奥行きが標高値へと変換できるため,DEMデータの構築とオルソモザイク画像の構築を行うことができる.本処理の処理結果を\fref{Z軸指定結果}に示す.

      \begin{figure}[t]
        \centering
        \includegraphics[width=9cm]{image/exmaple/view_reset.jpg}
        \caption{Z軸指定の処理結果}
        \label{Z軸指定結果}
      \end{figure}

    \subsection{DEM構築}
      \label{DEM構築}
      \ref{Z軸指定}項によって得られた直下視点の三次元モデルを投影することよって直下視の投影方向にてDEMデータの構築を行う.ここで,ソースデータに「高密度クラウド」を指定することで,より精度の高いDEMデータを構築することができる.一般的な測量手法では,地上基準点(GCP)やExifタグによりスケールや位置情報を埋め込むが,本研究でのヘリコプター空撮画像ではこれらが利用できないため,得られる標高値はモデル内での相対的な値となる.本処理の設定値を\fref{DEM構築設定値}に,\ref{Z軸指定}項の出力結果を用いた際の処理結果を\fref{DEM構築結果}に示す.
      また,処理名称は「DEM構築」であるが,本研究では植生や建物の標高値を含まない数値標高モデルのことをDEM,植生や建物の標高値を含む数値標高モデルのことをDSMと呼ぶ.

      \begin{figure}[t]
        \begin{minipage}[c]{0.45\hsize}
          \centering
          \includegraphics[width=4.5cm]{image/setting/dem.jpg}
          \subcaption{設定値}
          \label{DEM構築設定値}
        \end{minipage}
        \begin{minipage}[c]{0.45\hsize}
          \centering
          \includegraphics[width=9cm]{image/exmaple/dem.jpg}
          \subcaption{処理結果}
          \label{DEM構築結果}
        \end{minipage}
        \caption{DEM構築}
      \end{figure}

    \subsection{オルソモザイク構築}
      \label{オルソモザイク構築}
      \ref{DEM構築}項と同様,擬似的な直下視点画像であるオルソモザイク画像(以降,オルソ画像)を構築する.プロジェクション面に「現在のビュー」を指定し,オルソ画像を出力する.本処理の設定値を\fref{オルソモザイク構築設定値}に,\ref{Z軸指定}項の出力結果を用いた際の処理結果を\fref{オルソモザイク構築結果}に示す.

      \begin{figure}[t]
        \begin{minipage}[c]{0.45\hsize}
          \centering
          \includegraphics[width=4.5cm]{image/setting/ortho.jpg}
          \subcaption{設定値}
          \label{オルソモザイク構築設定値}
        \end{minipage}
        \begin{minipage}[c]{0.45\hsize}
          \centering
          \includegraphics[width=9cm]{image/exmaple/ortho.jpg}
          \subcaption{処理結果}
          \label{オルソモザイク構築結果}
        \end{minipage}
        \caption{オルソモザイク構築}
      \end{figure}


  \section{災害前DEMの前処理}
    \subsection{リサンプリング}
      \ref{土砂量推定}節にて災害前後の標高値差分を取る際に,災害前後の数値標高モデルの解像度が異なるためリサンプリング処理によって解像度を統一する.本研究では災害前の数値標高モデルに国土地理院提供のDEM,災害後の数値標高モデルは\ref{DEM構築}項にて出力したDSMを用いる.国土地理院提供のDEMのメッシュ解像度は5m,或いは10m,本研究で用いるDSMのメッシュ解像度は数十cm単位であるため,災害前数値標高モデルに対し最も滑らかな画素補間法であるバイキュービック補間\cite{バイキュービック法}を行い拡大することによって解像度を統一する.バイキュービック法は,以下の手順によって算出できる.
      
      \fref{ピクセル図}に示すような,注目画素の周囲$4\times4$(16)画素に対し,\Fref{バイキュービック法1}を適用することによって得られる.ただし,注目画素の座標を$(x,y)$,\Fref{バイキュービック法2},\Fref{バイキュービック法3}を満たし,$a$は一般的には$-1$とされる.

      \begin{eqnarray}
        \label{バイキュービック法1}
        i(x,y) = 
        \begin{pmatrix}
        h(x_{1}) & h(x_{2}) & h(x_{3}) & h(x_{4})
        \end{pmatrix}
        \begin{pmatrix}
          f_{11} & f_{12} & f_{13} & f_{14} \\
          f_{21} & f_{22} & f_{23} & f_{24} \\
          f_{31} & f_{32} & f_{33} & f_{34} \\
          f_{41} & f_{42} & f_{43} & f_{44}
        \end{pmatrix}
        \begin{pmatrix}
          h(y_{1}) \\ h(y_{2}) \\ h(y_{3}) \\ h(y_{4})
        \end{pmatrix}
      \end{eqnarray}
  
      \begin{eqnarray}
        \label{バイキュービック法2}
        \left\{
          \begin{array}{l}
            x_{1} = 1 + x - [x] \\
            x_{2} = x - [x]     \\
            x_{3} = [x] + 1 - x \\
            x_{4} = [x] + 2 - x \\
            y_{1} = 1 + y - [y] \\
            y_{2} = y - [y]     \\
            y_{3} = [y] + 1 - y \\
            y_{4} = [y] + 2 - y
          \end{array}
        \right.
      \end{eqnarray}

      \begin{eqnarray}
        \label{バイキュービック法3}
        h(t) = 
        \left\{
          \begin{array}{ll}
            (a + 2) |t|^3 - (a + 3)|t|^2 + 1 & (|t| \leq 1)   \\
            a|t|^3 - 5a|t|^2 + 8a|t| - 4a    & (1 < |t| \leq 2) \\
            0                                & (2 < |t|)
          \end{array}
        \right.
      \end{eqnarray}

      \begin{figure}[t]
        \centering
        \includegraphics[width=6cm]{image/method/bicubic.png}
        \caption{バイキュービック法におけるピクセル図}
        \label{ピクセル図}
      \end{figure}

    \subsection{災害前DEMの切り抜き}
      TODO: いらないかも,実験データ切り抜いて使うなら
      災害後空撮画像によって得られたDSMは国土地理院DEMに比べ空撮画像の撮影範囲分のみであり領域が狭いため,災害後DSMと同範囲を抽出することによって無駄な領域を削除する.この処理によって,後述の{災害後DSMの正規化}項において国土地理院DEMの該当領域内における最小・最大標高を基準とした正規化処理が可能となる.

    \subsection{傾斜角度・傾斜方位の算出}
    TODO: Hornの公式かも
      xxxの指標として,\ref{土砂移動推定}節にて,.....傾斜角度,傾斜方位を利用する.傾斜角度とは,ある画素の傾斜角度を表す指標であり,傾斜角度の値が大きいほど地表が急勾配であることを表す.傾斜方位とは,下りの傾斜角度が指している方向を表す\cite{傾斜角度・傾斜方位}.各指標は\Fref{傾斜角度},\Fref{傾斜方位}によって算出される.なお,傾斜方位は$0 \circ$から$360 \circ$(真北)で右回りに計算される指標である.$\dfrac{dz}{dx}$,$\dfrac{dz}{dy}$は$e$を中心とした\fref{メッシュ図}のような$3 \times 3$画素(9画素)のメッシュがあった場合に,中心画素の$X$方向と$Y$方向の変化率であり,\Fref{X方向の変化率},\Fref{Y方向の変化率}より算出される.\fref{入力DEM}のDEMを用いて導出した傾斜角度と傾斜方向を\fref{傾斜角度図}と\fref{傾斜方位図}に示す.なお,疑似カラーを用いて表しており,赤色が強いほど値が高く,青色が強いほど値が低いとする.

      \begin{equation}
        \label{傾斜角度}
        slope_{degrees} = 
        \dfrac{180} {\pi} \times 
        \arctan \sqrt{
          (\dfrac{dz}{dx})^2 + (\dfrac{dz}{dy})^2
        }
      \end{equation}

      \begin{equation}
        \label{傾斜方位}
        aspect = 
        \dfrac{180} {\pi} \times
        \arctan 2 (\dfrac{dz}{dy}, \dfrac{dz}{dx})
      \end{equation}

      \begin{equation}
        \label{X方向の変化率}
        \dfrac{dz}{dx} = 
          \dfrac {(c + 2f + i) - (a + 2d + g)} {8}
      \end{equation}

      \begin{equation}
        \label{Y方向の変化率}
        \dfrac{dz}{dy} = 
          \dfrac {(g + 2g + i) - (a + 2b + c)} {8}
      \end{equation}

      \begin{figure}[t]
        \centering
        \includegraphics[width=6cm]{image/method/mesh.png}
        \caption{傾斜角度・傾斜方位におけるメッシュ図}
        \label{メッシュ図}
      \end{figure}

      TODO: カラーランプ変えるかも
      \begin{figure}[t]
        \begin{minipage}[c]{0.3\hsize}
          \centering
          \includegraphics[width=4cm]{image/exmaple/dem_gsi.png}
          \subcaption{入力DEM}
          \label{入力DEM}
        \end{minipage}
        \begin{minipage}[c]{0.3\hsize}
          \centering
          \includegraphics[width=4cm]{image/exmaple/slope.png}
          \subcaption{傾斜角度図}
          \label{傾斜角度図}
        \end{minipage}
        \begin{minipage}[c]{0.3\hsize}
          \centering
          \includegraphics[width=4cm]{image/exmaple/aspect.png}
          \subcaption{傾斜方位図}
          \label{傾斜方位図}
        \end{minipage}
        \caption{傾斜角度・傾斜方位の算出}
      \end{figure}


  \section{領域データの抽出}
    本研究では,土砂領域検出及び建物領域検出の際に領域単位で処理を行うため,災害後オルソ画像より領域データを算出する.検出処理では主にピクセル単位,領域単位,オブジェクト単位等での検出単位が存在するが,本研究では色相情報を用いた分類を行うため,類似色相での領域単位で検出を行う.また,本節では各領域の面積,周囲長,輪郭座標を取得する.

    \subsection{オルソ画像の領域分割}
      \label{オルソ画像の領域分割}
      空撮画像等の土砂領域や植生領域を画素単位で検出することは難しいため,近傍画素との関係性を考慮した領域単位での判別を行う.本研究ではMean-Shift法\cite{Mean-Shift法1, Mean-Shift法2}を用いた領域分割を行う.Mean-Shift法とはカーネル密度推定によるクラスタリング手法の一つで,画像の領域分割,動画像における対象物体追跡に用いられる.また,領域分割の代表的な手法であるk-means法\cite{k-means法}に比べ,クラスタ数を事前に決める必要が無いという利点がある.Mean-Shift法は,$d$次元空間中の$N$個の点群を標本として得られるような確率密度関数$f(x)$を考え,その標本点から確率密度関数$f(x)$の極大点を探索する手法である.本研究では,Mean-Shift法による領域分割を行うことによって近傍の類似色画素を領域単位に分類しつつ,空撮画像の撮影機器や画像上の細かい地物による色や輝度のばらつきを抑制する.次に,Mean-Shift法にてカラー画像の領域分割を行う手順について説明する.

      \begin{enumerate}
        \setlength{\itemsep}{-5pt}
        \item カラー画像中の各画素の位置を二次元座標$x_i$,その画素値を三次元チャンネル$v_{i} =(R_{i},G_{i},B_{i})$とし,画素位置と画素値を結合した5次元空間内の点$z_{i} = (x_{i}, v_{i})$を考える.距離と色相が近い画素が5次元空間内でクラスタを成しているとし,各画素をMean-Shift法でクラスタリングする.
        \item すべての$z_{i}$にMean-Shift法を適用し,収束位置$z_{i}^c = (x_{i}^c , v_{i}^c)$を計算する.
        \item $x_{i}$の画素値を収束位置の画素の値$v^c = (R^c, G^c, B^c)$に置き換えることによって領域分割ができる.カーネル密度推定とMean-Shift法の計算式を\Fref{Mean-Shift法1}と\Fref{Mean-Shift法2}示す.ただし,\Fref{Mean-Shift法3}を満たすとする.
      \end{enumerate}
    
      \begin{equation}
        \label{Mean-Shift法1}
        f(x) = \dfrac{c} {N h_{s}^2 h_{r}^3}
          \sum_{i=1}^{N}
          k (|\dfrac{x^s - x_{i}^s} {h_{s}}|^2) k (|\dfrac{x^r - x_{i}^r} {h_{r}}|^2)
      \end{equation}

      \begin{equation}
        \label{Mean-Shift法2}
        \left\{
          \begin{array}{l}
            y_{j+1}^s = 
              \dfrac{\sum_{i=1}^{N} g_{i}^s x_{i}^s} {\sum_{i=1}^{N} g_{i}^s} \\ 
            y_{j+1}^r = 
              \dfrac{\sum_{i=1}^{N} g_{i}^r x_{i}^r} {\sum_{i=1}^{N} g_{i}^r}
          \end{array}
        \right.
      \end{equation}

      \begin{equation}
        \label{Mean-Shift法3}
        \left\{
          \begin{array}{l}
            g_{i}^s = k' (|\dfrac{y_{j}^s - x_{i}^s} {h_{s}}|^2)
              k  (|\dfrac{y_{i}^r - x_{i}^r} {h_{r}}|^2) \\
            g_{i}^r = k  (|\dfrac{y_{j}^s - x_{i}^s} {h_{s}}|^2) 
              k' (|\dfrac{y_{i}^r - x_{i}^r} {h_{r}}|^2)
          \end{array}
        \right.
      \end{equation}

      なお,本研究ではMean-Shift法の特徴量空間に距離を表す画素位置$(x,y)$,色相を表す画素値$(R,G,B)$を用いるため5次元空間での処理となり,距離・色相の近い画素群が一つの領域となる.\ref{オルソモザイク構築}項で得られたオルソ画像の一部を切り取った画像を\fref{Mean−Shift法入力}に,Mean-Shift法を適用した結果を\fref{Mean-Shift法出力}に示す.

      \begin{figure}[t]
        \begin{minipage}[c]{0.45\hsize}
          \centering
          \includegraphics[width=\linewidth]{image/exmaple/trimming.png}
          \subcaption{入力画像}
          \label{Mean−Shift法入力}
        \end{minipage}
        \begin{minipage}[c]{0.45\hsize}
          \centering
          \includegraphics[width=\linewidth]{image/exmaple/mean-shift.png}
          \subcaption{出力画像}
          \label{Mean−Shift法出力}
        \end{minipage}
        \caption{Mean-Shift法}
      \end{figure}

      また,この際に各領域の座標データをコンピュータ上に保存する...(良い書き方ねえか??)ファイル,テキストファイル,バイナリファイル,イメージデータとして

    \subsection{領域データの算出}

    \subsection{or カラーラベリング}
      通常のラベリング処理では,類似色の領域が潰れてしまうという欠点がある.
      \ref{オルソ画像の領域分割}項にて取得した座標データを利用し,ラベリングテーブルを作成する.作成したラベリングテーブルより,領域番号毎に輪郭抽出を行う...
      領域毎に以下の処理を行う.
      \begin{enumerate}
        \setlength{\itemsep}{-5pt}
        \item 領域の全座標を取得
        \item 領域座標よりを用いてその領域の輪郭抽出を行う
        \item 輪郭データより面積・周囲長頭を算出
        \item ラベリングテーブルに保存...みたいな?
      \end{enumerate}
      


  \section{災害後DSMの前処理}
    DSMは建物・樹木等を含む数値標高モデルに対し,DEMは建物・樹木等の地表物の高さを含まないため,単純に災害前後の標高値差分を取った場合にずれが生じる.よって,ここでは災害後DSM全体及び建物領域に対し標高値補正を行う.植生領域については,\ref{植生除去}項にて除去を行う.

    \subsection{ジオリファレンサ}
      \ref{入力データ}節で用いる空撮画像において,本研究では位置情報が埋め込まれていない画像群の例が存在する.この場合,\ref{オルソモザイク構築}項,\ref{DEM構築}項で前述したようにオルソ画像や災害後DSMは位置情報を持たない.しかし,後述の処理でオルソ画像や災害後DSMと,国土地理院DEM等の位置情報が対応付けられいてる必要がある.そのため,災害後DSMに位置情報が付与されていない場合,オープンソースの地理情報システム\footnote{GIS(Geographic Information System)}であるQGIS\cite{QGIS}のジオリファレンサ機能を利用して位置合わせを行う.ジオリファレンサとは,位置情報を持たないラスタデータ\footnote{画像同様に格子状のピクセル構造を持つデータ構造}に任意の位置情報を付与する手法であり,この処理によって位置情報を持たないオルソ画像や災害後DSMを地図上に重畳でき,国土地理院DEM等と位置情報の対応付けが可能となる.

      QGISにてジオリファレンサを起動し,ラスタデータと地図上のランドマークの対応点付けを行う.ラスタデータのランドマークを選択し,地図上の同一のランドマークを選択する.最低4箇所の対応付けが必要であり,ラスタデータと地図上の対応するランドマークの対応付け箇所が多いほど精度が向上する.本処理の設定値を\fref{ジオリファレンサ設定値}に,\ref{DEM構築}項の出力結果を国土地理院の標準地図\cite{標準地図}上に重畳した結果を\fref{ジオリファレンサ結果}に示す.

      \begin{figure}[t]
        \begin{minipage}[c]{0.45\hsize}
          \centering
          \includegraphics[width=\linewidth]{image/setting/georeferencer.png}
          \subcaption{設定値}
          \label{ジオリファレンサ設定値}
        \end{minipage}
        \begin{minipage}[c]{0.45\hsize}
          \centering
          \includegraphics[width=\linewidth]{image/exmaple/georeferencer.png}
          \subcaption{DSMの重畳結果}
          \label{ジオリファレンサ結果}
        \end{minipage}
        \caption{ジオリファレンサ}
      \end{figure}

    \subsection{建物領域の検出}
      \label{建物領域の検出}
      TODO: 凸領域で検出したかった
      建物領域は直下視点の空撮画像中では単純な形状であることが多い.そのため,\ref{オルソモザイク構築}節で得たオルソ画像の各領域に対し円形度による閾値処理を行うことによって,建物候補領域を抽出する.円形度は領域形状の簡易さを表す指標であり,\Fref{円形度}で表される.$S$は領域の面積,$L$は領域の周囲長,$C$は円形度を示す.

      \begin{equation}
        \label{円形度}
        C = \dfrac{4 \pi S} {L^2} 
      \end{equation}

      その後,テクスチャ特徴による分類を行うことによって建物領域を抽出する.建物領域の屋根部分は均一であるという特徴を持つため,ある領域内の画素の不均一性を示す指標である異質度(dissimilarity)\cite{異質度}を導入する.異質度\Fref{異質度}による閾値処理を行うことによって,建物領域を抽出する.$P(i,j)$は$(i,j)$における画素値,$dissimilarity$は異質度を示す.

      TODO: もう少し詳細に(使う場合)
      \begin{equation}
        \label{異質度}
        dissimilarity = \sum_{i=0}^{255} \sum_{j=0}^{255} (i,j) |i-j|
      \end{equation}

      TODO: QGISでの処理も書くかも
      また,国土地理院の提供する建物ポリゴンを利用することによって,建物候補領域から建物領域を検出する.建物候補領域の検出結果に建物ポリゴンを重畳し,建物候補領域に建物ポリゴンのピクセルが50\%以上含まれる場合,その建物候補領域を建物領域として検出する.\fref{オルソ画像の領域分割}項の入力画像より建物領域を検出した例を\fref{建物領域検出}に示す.

      \begin{figure}[t]
        \centering
        \includegraphics[width=9cm]{image/exmaple/building.png}
        \caption{建物領域の検出結果}
        \label{建物領域検出}
      \end{figure}

    \subsection{建物領域の標高値補正}
    TODO: もうちょい詳しく(他の手法使わないなら)
      その後,抽出した建物領域に隣接する地表面領域の標高値を用いて建物領域の全画素の標高値を地表面領域の平均標高値に変換することで,建物領域の標高値を除去する.

    \subsection{災害後DSMの正規化}
      \label{災害後DSMの正規化}
      次に,\ref{DEM構築}によって得られたDSMはモデル内での相対的な標高値であるため,国土地理院DEMと水平位置を対応付けし,\Fref{正規化}を適用することによって正規化した絶対的な標高値を与える.$x$は正規化前のDSMの標高値,$x'$は正規化後のDSMの標高値を示す.

      \begin{equation}
        \label{正規化}
        x'_{i} = \dfrac{x_{i} - min(x)} {max(x) - min(x)} \times max(DEM) 
      \end{equation}


  \section{土砂領域のマスク画像作成}
    \label{土砂マスク}
      TODO: ヒストグラム平坦化とかするかも

    \subsection{L*a*b*表色系への変換}
    TODO: 文献4は一つめの引用にした方が良いかも?

      オルソ画像に対し L*a*b*変換\cite{Lab表色系1}を行う.L*a*b*表色系とは輝度をL*,色相と彩度を示す色度をa*,b*で表した表色系であり,人間の視覚に近い表色系であるため色情報を用いて分類を行う際に有効な指標であることが示されている.また,空撮画像中の土砂領域はRGB表色系における赤色の値が大きく,植生領域は緑色の値が大きいという特徴があるため,本研究ではL*a*b*表色系を用いて土砂領域及び植生領域を検出する\cite{Lab表色系2, Lab表色系3, Lab表色系4}.本研究で用いる画像は本処理時点でRGB表色系にて表されているためL*a*b*表色系への変換を行う.RGB表色系からL*a*b*表色系への変換式を\Fref{Lab表色系1}と\Fref{Lab表色系2}に示す.RGB表色系はデバイス依存色であり直接L*a*b*表色系に変換する式は存在しないため,デバイス独立色であるXYZ表色系\cite{XYZ表色系}に変換してから処理を行う.XYZ表色系への変換式を\Fref{XYZ表色系}に示す.$X,Y,Z$,$R,G,B$はそれぞれ,XYZ表色系及びRGB表色系の各チャンネルの画素値を表す.また,上記指標のうち,\ref{オルソ画像の領域分割}項の領域分割画像において検出に用いるa*値,b*値画像の例を\fref{Lab画像}に示す.

      \begin{eqnarray}
      \label{Lab表色系1}
        \left\{
          \begin{array}{l}
            L^* = 116 \times f(Y^{\frac{1}{3}}) - 16 \\
            a^* = 500 \times (f(X) - f(Y)) \\
            b^* = 200 \times (f(Y) - f(Z))
          \end{array}
        \right.
      \end{eqnarray}

      \begin{eqnarray}
        \label{Lab表色系2}
          f(t) = 
          \left\{
            \begin{array}{lll}
              \sqrt[3]{t} 
                &(t >    (\dfrac{6} {29})^3) \\
              \dfrac{1} {3} (\dfrac{29} {6})^3 t + \dfrac{4} {29}
                &(t \leq (\dfrac{6} {29})^3)
            \end{array}
          \right.
      \end{eqnarray}

      \begin{eqnarray}
        \label{XYZ表色系}
        \left\{
          \begin{array}{l}
            X = 0.412453R + 0.357580G + 0.180423B \\
            Y = 0.212671R + 0.715160G + 0.072169B \\
            Z = 0.019334R + 0.119193G + 0.950227B
          \end{array}
        \right.
      \end{eqnarray}
      
      \begin{figure}[t]
        \begin{minipage}[c]{0.45\hsize}
          \centering
          \includegraphics[width=\linewidth]{image/exmaple/lab-a.png}
          \subcaption{a*値画像}
        \end{minipage}
        \begin{minipage}[c]{0.45\hsize}
          \centering
          \includegraphics[width=\linewidth]{image/exmaple/lab-b.png}
          \subcaption{b*値画像}
        \end{minipage}
        \caption{L*a*b*表色系の画像例}
        \label{Lab画像}
      \end{figure}

    \subsection{土砂領域検出}
      土砂災害は植生の多い山間部の斜面にて多発し,土砂領域を検出する際に誤検出する可能性がある.よって,植生領域を検出結果より除去する.土砂領域,植生領域は\tref{領域特徴}に示す特徴を持つためL*a*b*表色系のa*値,b*値の指標を用いた閾値処理を行い,土砂候補領域,植生領域を検出する.検出した土砂候補領域より植生領域を除去した領域を土砂領域とする.{オルソ画像の領域分割}項の入力画像より土砂領域を検出した結果を\fref{土砂領域検出}に示す.

    TODO: 土砂マスクでこの指標で試す

    \begin{table}[b]
      \centering
      \caption{領域特徴}
      \label{領域特徴}
      \begin{tabular}{ccc}
        \hline
        領域 & a* & b* \\ \hline
      \end{tabular}
    \end{table}
    
    \begin{figure}[t]
      \begin{minipage}[c]{0.3\hsize}
        \centering
        \includegraphics[width=4cm]{image/exmaple/sediment_candidate.png}
        \subcaption{土砂候補領域の検出結果}
      \end{minipage}
      \begin{minipage}[c]{0.3\hsize}
        \centering
        \includegraphics[width=4cm]{image/exmaple/vegitation.png}
        \subcaption{植生領域の検出結果}
      \end{minipage}
      \begin{minipage}[c]{0.3\hsize}
        \centering
        \includegraphics[width=4cm]{image/exmaple/sediment.png}
        \subcaption{土砂領域の検出結果}
      \end{minipage}
      \caption{土砂領域の検出結果}
      \label{土砂領域検出}
    \end{figure}

    \subsection{土砂領域のマスク画像作成}
    TODO: 傾斜度で30°以上の領域だけに限定するかも

      {土砂領域検出}項の検出結果より,二値の土砂マスク画像を作成する.作成する土砂マスク画像の目的として植生領域を除去し,後の土砂量と土砂移動推定にて範囲を限定する目的で使用する.そのため,本研究では土砂領域中の建物領域やその他の地物等は土砂マスク画像の土砂領域として検出する.\ref{土砂領域}の検出結果は色相情報を用いているため屋根の色相によっては建物領域を未検出とすることがある.よって,建物領域の穴を埋め土砂領域として扱うため,クロージング処理\cite{クロージング処理}を適用した後,面積が閾値以下の小領域を除去する.
      
      クロージング処理とは二値画像において膨張処理\footnote{注目画素の8近傍に白画素がある場合,注目画素を白色にする処理}を複数回行った後,収縮処理\footnote{注目画素の8近傍に黒画素がある場合,注目画素を黒色にする処理}を同回数行い,画像中の穴を除去する処理である.\ref{土砂領域検出}項の検出結果の二値画像におけるクロージング処理の適用例を\fref{クロージング処理}に示す.

      \begin{figure}[t]
        \begin{minipage}[c]{0.45\hsize}
          \centering
          \includegraphics[width=\linewidth]{image/exmaple/lab-a.png}
          \subcaption{入力画像}
        \end{minipage}
        \begin{minipage}[c]{0.45\hsize}
          \centering
          \includegraphics[width=\linewidth]{image/exmaple/lab-b.png}
          \subcaption{出力画像}
        \end{minipage}
        \caption{クロージング処理の出力結果}
        \label{クロージング処理}
      \end{figure}

      TODO: ラベリングの引用か注釈つける
      この後,二値画像に対しラベリング処理\cite{}を行うことによって各領域の面積を算出する.面積における閾値処理によって,一般住宅と同程度,或いはそれ以下の面積の小領域を除去する.先程のクロージング処理結果に対し小領域除去を行った結果を\fref{小領域除去}に示す.
      
      \begin{figure}[t]
        \centering
        \includegraphics[width=9cm]{image/exmaple/building.png}
        \caption{小領域の除去結果}
        \label{小領域除去}
      \end{figure}


  \section{土砂量推定}
    \label{土砂量推定}
    災害後の空撮画像より作成した標高値補正済みDSMと災害前の解像度補正済み国土地理院DEMの被災前後での標高値差分を求めることにより,災害前後での土砂量変化を算出する.また,\ref{土砂マスク}節で作成した土砂マスクの領域に限定して処理を行うことにより,植生領域のずれや誤検出を除去する.


  \section{土砂移動推定}
    \label{土砂移動推定}
      TODO: 開閉地形等を基準の中心点とできないか

    精度評価の際にMean-Shift法での領域ベースでの正解データの目視判読による作成は領域数の多さに起因し困難であるため,メッシュベースでの土砂移動推定を行う.今回は地表面での土砂移動が確認できたランドマークの最大移動距離に合わせ,オルソ画像を縦13メッシュ,横12メッシュに区切った.

    土砂移動は一般的に上流から下流へ流下する.また,一般的に災害後には上流領域は土砂崩壊部で侵食が発生し,下流領域には上流より流下した土砂が堆積する.よって,メッシュ中心画素から以下の4条件を満たす画素を全て追跡し,メッシュ中心画素から最も距離の遠い画素へ移動線を結ぶことによって土砂移動が発生したと推定する.

    ・	上流方向から下流方向に沿って傾斜方向が向いている

    ・	上流画素の標高値が下流画素の標高値より低い
    
    ・	画素同士が隣接している
    
    ・	侵食領域と堆積領域の組み合わせである

    傾斜方向の判別には災害後DSMより作成した傾斜方向モデルを用い,侵食領域と堆積領域の組み合わせの判別には\ref{土砂量推定}節の土砂量推定結果を用いる.
  

