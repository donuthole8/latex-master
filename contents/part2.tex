\chapter{手法}
  \section{概要}
    提案手法概要図を\fref{img2-1}に示す.まず,災害後空撮画像の三次元復元を行うことによってオルソ画像(直下視点画像)とDSM(数値標高モデル)の作成を......

    提案手法では,入力として災害後空撮画像,国土地理院DEMを利用し,土砂量図と土砂移動図を出力とする.


    \begin{figure}[t]
      \centering
      \includegraphics[width=12cm]{image/processing-flow.png}
      \caption{提案手法概要図}
      \label{img2-1}
    \end{figure}


  \section{入力データ}
    入力画像の特徴等\dots



  \section{三次元復元}
    まず,災害後空撮画像の三次元復元を行うことによってオルソ画像(直下視点画像)とDSM(数値標高モデル)の作成を行う.三次元復元とは複数枚画像を用いて被写体の形状や距離等の3次元情報を復元する処理である.オルソ画像は,三次元復元によって得られた三次元モデルを真上から投影した状態の各画素を取得することによって得る.DSMは三次元モデルを真上から投影した状態の数値標高モデルの各標高値を取得することによって得る.本手法では復元精度の高いAgisoft社のMetashape\cite{使用手法1}を利用する.

    その後,災害後空撮画像によって得られたDSMは国土地理院DEMに比べ空撮画像の撮影範囲分のみであり領域が狭いため,災害後DSMと同範囲を抽出することによって無駄な領域を削除する.この処理によって,後述の災害後DSMの標高値の正規化処理において国土地理院のDEMの最大標高を基準とした正規化処理が可能となる.
  
    また,災害前後での標高値モデルは解像度が異なるため,最も滑らかな画素補間手法であるバイキュービック法\cite{論文手法1}を用いることによって解像度を統一する.一般的には国土地理院DEMは解像度が粗いため,この手法を用い国土地理院DEMを拡大する.
    
    最後に,後述の処理で色相と水平座標の距離が近い領域単位での処理を行うため,オルソ画像に対しMean-Shift法\cite{論文手法2}による領域分割を行う.また,この処理によって空撮画像の撮影機器や画像上の細かい地物による色や輝度のばらつきを抑制する.

    TODO: Mean-Shift法書く


    \subsection{写真のアライメント}
    \subsection{高密度クラウドの構築}
    \subsection{メッシュの構築}
    \subsection{テクスチャの構築}
    \subsection{オルソ画像の作成}
      \label{オルソ画像}
    \subsection{DSMの作成}




  \section{災害前DEMの前処理}
    \subsection{リサンプリング}
      災害後DSM領域の抽出も一緒に

    \subsection{災害前DEMの正規化}
      次に,三次元復元によって得られたDSMは基準となる標高値を持たないため,国土地理院DEMと水平位置を対応付け,\Fref{正規化}を適用することによって標高値を正規化する.$x$は正規化前のDSMの標高値,$x'$は正規化後のDSMの標高値を示す.

      \begin{equation}
        \label{正規化}
        x'_{i} = \dfrac{x_{i} - min(x)} {max(x) - min(x)} \times max(DEM) 
      \end{equation}
  
    \subsection{災害前DEMの切り抜き}

    \subsection{傾斜度データの作成}
    \subsection{傾斜方位データの作成}

  \section{領域データの抽出}
    \subsection{オルソ画像の領域分割}
    \subsection{カラーラベリング}



  \section{建物領域の標高値補正}
    \subsection{建物領域の検出}
    

      DSMは建物・樹木等を含む標高値であることに対し,DEMは建物・樹木等の地表物の高さを含まないため,単純に災害前後の標高値差分を取った場合にずれが生じる.よって,ここでは建物領域の標高値補正を行う.植生領域については,\ref{植生除去}項にて除去を行う.

      \ref{オルソ画像}節で得たオルソ画像の各領域に対し円形度\Fref{円形度}による閾値処理を行うことによって,建物候補領域を抽出する.$S$は領域の面積,$L$は領域の周囲長,$C$は円形度を示す.

      \begin{equation}
        \label{円形度}
        C = \dfrac{4 \pi S} {L^2} 
      \end{equation}

      その後,テクスチャ特徴による分類を行うことによって建物領域を抽出する.建物領域の屋根部分は均一であるという特徴を持つため,ある領域内の画素の不均一性を示す指標である異質度(dissimilarity)\cite{論文手法3}を導入する.異質度\Fref{異質度}による閾値処理を行うことによって,建物領域を抽出する.$P(i,j)$は$(i,j)$における画素値,$dissimilarity$は異質度を示す.

      \begin{equation}
        \label{異質度}
        dissimilarity = \sum_{i=0}^{255} \sum_{j=0}^{255} (i,j) |i-j|
      \end{equation}

      その後,抽出した建物領域に隣接する地表面領域の標高値を用いて建物領域の全画素の標高値を地表面領域の平均標高値に変換することで,建物領域の標高値を除去する.
      あと国土地理院マスク..

    \subsection{隣接地表面領域の抽出}

  
  \section{土砂領域のマスク画像作成}
    \label{土砂マスク}
    \subsection{土砂候補領域検出}
      オルソ画像中の土砂領域・植生領域はそれぞれ赤色味・緑色味が強くなる傾向がある.本研究では人間の色覚に類似したL*a*b*表色系を用い,赤色味の強い画素では$a*$値が高く,緑色味の強い画素では $a*$値が低いという特徴を利用した閾値処理によって土砂候補領域・植生領域の検出を行う.
      
      検出した土砂候補領域より植生領域を除去し,土砂領域として残った領域を二値化することによって土砂マスクを生成する.
  
    \subsection{植生領域検出}
      \label{植生除去}
    \subsection{統合処理による土砂領域検出}
    \subsection{土砂領域検出結果の二値化処理}
    \subsection{土砂領域のマスク画像作成}

  
  \section{土砂量推定}
    \label{土砂量推定}
    災害後の空撮画像より作成した標高値補正済みDSMと災害前の解像度補正済み国土地理院DEMの被災前後での標高値差分を求めることにより,災害前後での土砂量変化を算出する.また,\ref{土砂マスク}節で作成した土砂マスクの領域に限定して処理を行うことにより,植生領域のずれや誤検出を除去する.
  

  \section{土砂移動推定}
    \label{土砂移動推定}
    精度評価の際にMean-Shift法での領域ベースでの正解データの目視判読による作成は領域数の多さに起因し困難であるため,メッシュベースでの土砂移動推定を行う.今回は地表面での土砂移動が確認できたランドマークの最大移動距離に合わせ,オルソ画像を縦13メッシュ,横12メッシュに区切った.

    土砂移動は一般的に上流から下流へ流下する.また,一般的に災害後には上流領域は土砂崩壊部で侵食が発生し,下流領域には上流より流下した土砂が堆積する.よって,メッシュ中心画素から以下の4条件を満たす画素を全て追跡し,メッシュ中心画素から最も距離の遠い画素へ移動線を結ぶことによって土砂移動が発生したと推定する.

    ・	上流方向から下流方向に沿って傾斜方向が向いている

    ・	上流画素の標高値が下流画素の標高値より低い
    
    ・	画素同士が隣接している
    
    ・	侵食領域と堆積領域の組み合わせである

    傾斜方向の判別には災害後DSMより作成した傾斜方向モデルを用い,侵食領域と堆積領域の組み合わせの判別には\ref{土砂量推定}節の土砂量推定結果を用いる.
  

