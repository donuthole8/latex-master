\chapter{手法}
  \section{概要}
    提案手法概要図を\fref{提案手法概要図}に示す.まず,災害後空撮画像の三次元復元を行うことによってオルソ画像(直下視点画像)とDSM(数値標高モデル)の作成を......

    提案手法では,入力として災害後空撮画像,国土地理院DEMを利用し,土砂量図と土砂移動図を最終出力結果とする.

    \begin{figure}[t]
      \centering
      \includegraphics[width=12cm]{image/processing-flow.png}
      \caption{提案手法概要図}
      \label{提案手法概要図}
    \end{figure}


  \section{入力データ}
    \label{入力データ}
    入力データの特徴等\dots

    - 複数枚の災害後空撮画像,直下視点が望ましい,季節は植生が緑が良いかも,災害前DEMが整備されている,視差が得られるように,

    - 災害前DEM,5mメッシュが望ましい,



  \section{三次元復元}
    まず,災害後空撮画像の三次元復元を行うことによってオルソ画像(直下視点画像)とDSM(数値標高モデル)の作成を行う.三次元復元とは複数枚画像を用いて被写体の形状や距離等の3次元情報を復元する処理である.オルソ画像は,三次元復元によって得られた三次元モデルを真上から投影した状態の各画素を取得することによって得る.DSMは三次元モデルを真上から投影した状態の数値標高モデルの各標高値を取得することによって得る.本手法では復元精度の高いAgisoft社のMetashape\cite{使用手法1}を利用する.

    その後,災害後空撮画像によって得られたDSMは国土地理院DEMに比べ空撮画像の撮影範囲分のみであり領域が狭いため,災害後DSMと同範囲を抽出することによって無駄な領域を削除する.この処理によって,後述の災害後DSMの標高値の正規化処理において国土地理院のDEMの最大標高を基準とした正規化処理が可能となる.
  
    また,災害前後での標高値モデルは解像度が異なるため,最も滑らかな画素補間手法であるバイキュービック法\cite{論文手法1}を用いることによって解像度を統一する.一般的には国土地理院DEMは解像度が粗いため,この手法を用い国土地理院DEMを拡大する.

    最後に,後述の処理で色相と水平座標の距離が近い領域単位での処理を行うため,オルソ画像に対しMean-Shift法\cite{論文手法2}による領域分割を行う.また,この処理によって空撮画像の撮影機器や画像上の細かい地物による色や輝度のばらつきを抑制する.


    TODO: アラインメント?アライメント?使ってるMetashapeの名称に合わせる
    \subsection{写真のアラインメント}
      \label{写真のアラインメント}
      入力画像に対して特徴点の検出を行い,特徴マッチングを行うことで疎な点群の復元を行う.植生や水域等の特徴の少ない領域が原因でマッチングに失敗する場合があるため,失敗した画像を除去し再度処理を行う.また,精度を指定することが可能であり,精度を下げると入力画像の解像度を下げることによって処理時間を短縮する.提案手法では「高」(最大精度....最高かも)を指定し,その他の項目では既定値にて処理を行う.本処理の設定値を\fref{}に,\ref{入力データ}節を用いた時の処理結果を\fref{}に示す.

    \subsection{高密度クラウド構築}
      \label{高密度クラウド構築}
      \ref{写真のアラインメント}項によって得られた点群と入力画像より法線を算出し,密な点群を生成する.精度を指定することが可能であり,後述の\ref{DEM構築}項の精度に影響する.また,深度フィルタによって外れ値の除去を行うことができる.提案手法では「高」に設定し,その他の項目では既定値にて処理を行う.本処理の設定値を\fref{}に,\ref{写真のアラインメント}項の処理結果を用いた時の処理結果を\fref{}に示す.

    \subsection{メッシュ構築}
      \label{メッシュ構築}
      \ref{高密度クラウド構築}項によって得られた点群からポリゴンメッシュの構築を行う.三次元点群から三次元モデルを復元することによって,後述の\ref{Z軸指定}項の処理が用意になる.本処理の設定値を\fref{}に,\ref{高密度クラウド}項の処理結果を用いた時の処理結果を\fref{}に示す.

    \subsection{テクスチャ構築}
      \label{テクスチャ構築}
      \ref{メッシュ構築}項によって生成元の点群の色が割り当てられるが,頂点の色のみが対象であるため解像度が低い.そのため,本処理によって入力画像のテクスチャを生成したメッシュに貼り付ける.本処理の設定値を\fref{}に,\ref{メッシュ構築}項の処理結果を用いた時の処理結果を\fref{}に示す.

    \subsection{Z軸指定}
      \label{Z軸指定}
      おれのは直下視点だから別話,,,

      やり方書く(変な処理だったら消して良いかも〜)
      本処理の処理結果を\fref{}に示す.




      ここまでで構築した三次元モデルが直下視となるよう,Metashape 上にて回転を行う.ヘリコプター撮影映像が鳥瞰視点であるため,正しい Z 軸を Metashape 上で推定することは困難である. したがって本研究では,手動による回転を行う.
      これにより奥行きが標高値へと変換できるため,DEMデータの構築とオルソモザイク画像の構築を行うことができる.処理結果を図2.9に示す

    \subsection{オルソモザイク構築}
      \label{オルソモザイク構築}
      \ref{テクスチャ構築}項 or \ref{Z軸指定}項によって得られた直下視点の三次元モデルを投影することよって擬似的な直下視点画像(オルソモザイク画像,以降,オルソ画像)を構築する.プロジェクション面に「現在のビュー」を指定し,しオルソ画像を出力する.本処理の設定値を\fref{}に,\ref{Z軸指定}項の処理結果を用いた際の処理結果を\fref{}に示す.
      
    \subsection{DEM構築}
      \label{DEM構築}
      \ref{オルソモザイク構築}項と同様,プロジェクション面に「現在のビュー」を指定することで直下視の投影方向にてDEMデータの構築を行う.ここで,ソースデータに「高密度クラウド」を指定することで,より精度の高いDEMデータを構築することができる.一般的な測量手法では,地上基準点(GCP)やExifタグによりスケールや位置情報を埋め込むが,本研究でのヘリコプター空撮画像ではこれらが利用できないため,得られる標高値はモデル内での相対的な値となる.本処理の設定値を\fref{}に,\ref{Z軸指定}項の処理結果を用いた際の処理結果を\fref{}に示す.
      また,処理名称は「DEM構築」であるが,本研究では植生や建物の標高値を含まない標高値モデルのことをDEM(Digital Elevation...,どっかで前述してあれば良い,多分入力データでしてる),植生や建物の標高値を含む標高値モデルのことをDSM(Digital Surface Model)と呼ぶ.


  \section{災害前DEMの前処理}
    \subsection{リサンプリング}
      災害後DSM領域の抽出も一緒に

    \subsection{災害前DEMの正規化}
      次に,三次元復元によって得られたDSMは基準となる標高値を持たないため,国土地理院DEMと水平位置を対応付け,\Fref{正規化}を適用することによって標高値を正規化する.$x$は正規化前のDSMの標高値,$x'$は正規化後のDSMの標高値を示す.

      \begin{equation}
        \label{正規化}
        x'_{i} = \dfrac{x_{i} - min(x)} {max(x) - min(x)} \times max(DEM) 
      \end{equation}
  
    \subsection{災害前DEMの切り抜き}

    \subsection{傾斜度データの作成}
    \subsection{傾斜方位データの作成}

  \section{領域データの抽出}
    \subsection{オルソ画像の領域分割}
      空撮画像等の土砂領域や植生領域を画素単位で検出することは難しいため,近傍画素との関係性を考慮した領域単位での判別を行う.本研究ではMean-Shift法\cite{}を用いた領域分割を行う.Mean-Shift法はカーネル密度推定によるクラスタリング手法の一つで,画像の領域分割,動画像における対象物体追跡に用いられる.また,領域分割の代表的な手法であるk-means法\cite{}に比べ,クラスタ数を事前に決める必要が無いという利点がある.Mean-Shift法は,$d$次元空間中の$N$個の点群を標本として得られるような確率密度関数$f(x)$を考え,その標本点から確率密度関数$f(x)$の極大点を探索する手法である.次に,Mean-Shift法にてカラー画像の領域分割を行う手順について説明する.

      1. カラー画像中の各画素の位置を二次元座標$x_i$,その画素値を三次元チャンネル$v_{i} =(R_{i},G_{i},B_{i})$とし,画素位置と画素値を結合した5次元空間内の点$z_{i} = (x_{i}, v_{i})$を考える.距離と色相が近い画素が5次元空間内でクラスタを成しているとし,各画素をMean-Shift法でクラスタリングする.

      2. すべての$z_{i}$にMean-Shift法を適用し,収束位置$z_{i}^c = (x_{i}^c , v_{i}^c)$を計算する.
      
      3. $x_{i}$の画素値を収束位置の画素の値$v^c = (R^c, G^c, B^c)$に置き換えることによって領域分割ができる.カーネル密度推定とMean-Shift法の計算式を\Fref{Mean-Shift法1}と\Fref{Mean-Shift法2}示す.ただし,\Fref{Mean-Shift法3}を満たすとする.
    
      \begin{equation}
        \label{Mean-Shift法1}
        f(x) = \dfrac{c} {N h_{s}^2 h_{r}^3}
          \sum_{i=1}^{N}
          k (|\dfrac{x^s - x_{i}^s} {h_{s}}|^2) k (|\dfrac{x^r - x_{i}^r} {h_{r}}|^2)
      \end{equation}

      TODO: 2行目,最後のr,式あってるか確認
      \begin{equation}
        \label{Mean-Shift法2}
        y_{j+1}^s = 
          \dfrac{\sum_{i=1}^{N} g_{i}^s x_{i}^s} {\sum_{i=1}^{N} g_{i}^s}, 
        y_{j+1}^r = 
          \dfrac{\sum_{i=1}^{N} g_{i}^r x_{i}^r} {\sum_{i=1}^{N} g_{i}^r??}
      \end{equation}

      \begin{equation}
        \label{Mean-Shift法3}
        g_{i}^s = k' (|\dfrac{y_{j}^s - x_{i}^s} {h_{s}}|^2)
                  k  (|\dfrac{y_{i}^r - x_{i}^r} {h_{r}}|^2), 
        g_{i}^r = k  (|\dfrac{y_{j}^s - x_{i}^s} {h_{s}}|^2) 
                  k' (|\dfrac{y_{i}^r - x_{i}^r} {h_{r}}|^2)
      \end{equation}

      なお,本研究ではMean-Shift法の特徴量空間に距離を表す画素位置$(x,y)$,色相を表す画素値$(R,G,B)$を用いるため5次元空間での処理となり,距離・色相の近い画素群が一つの領域となる.領域分割の適用例を\fref{}に示す

    \subsection{カラーラベリング}



  \section{建物領域の標高値補正}
    \subsection{建物領域の検出}
    

      DSMは建物・樹木等を含む標高値であることに対し,DEMは建物・樹木等の地表物の高さを含まないため,単純に災害前後の標高値差分を取った場合にずれが生じる.よって,ここでは建物領域の標高値補正を行う.植生領域については,\ref{植生除去}項にて除去を行う.

      \ref{オルソモザイク構築}節で得たオルソ画像の各領域に対し円形度\Fref{円形度}による閾値処理を行うことによって,建物候補領域を抽出する.$S$は領域の面積,$L$は領域の周囲長,$C$は円形度を示す.

      \begin{equation}
        \label{円形度}
        C = \dfrac{4 \pi S} {L^2} 
      \end{equation}

      その後,テクスチャ特徴による分類を行うことによって建物領域を抽出する.建物領域の屋根部分は均一であるという特徴を持つため,ある領域内の画素の不均一性を示す指標である異質度(dissimilarity)\cite{論文手法3}を導入する.異質度\Fref{異質度}による閾値処理を行うことによって,建物領域を抽出する.$P(i,j)$は$(i,j)$における画素値,$dissimilarity$は異質度を示す.

      TODO: もう少し詳細に
      \begin{equation}
        \label{異質度}
        dissimilarity = \sum_{i=0}^{255} \sum_{j=0}^{255} (i,j) |i-j|
      \end{equation}

      その後,抽出した建物領域に隣接する地表面領域の標高値を用いて建物領域の全画素の標高値を地表面領域の平均標高値に変換することで,建物領域の標高値を除去する.
      あと国土地理院マスク..

    \subsection{隣接地表面領域の抽出}

  
  \section{土砂領域のマスク画像作成}
    \label{土砂マスク}
    \subsection{L*a*b*表色系}
    斜面崩壊・浸水領域での色相や輝度などの特徴を表2.1に示す.これらの特徴を用いて各領域の検出を行う.x〜x項に本手法で用いる指標の算出方法と各災害領域について示す.

      オルソ画像中の土砂領域・植生領域はそれぞれ赤色味・緑色味が強くなる傾向がある.本研究では人間の色覚に類似したL*a*b*表色系を用い,赤色味の強い画素では$a*$値が高く,緑色味の強い画素では $a*$値が低いという特徴を利用した閾値処理によって土砂候補領域・植生領域の検出を行う.
      
      検出した土砂候補領域より植生領域を除去し,土砂領域として残った領域を二値化することによって土砂マスクを生成する.
  

      表\tref{}の特徴に従って分類を行うためヒストグラム均一化を行った画像に対しL*a*b*変換[8]を行う.L*a*b*色空間とは輝度を$L*$,色相と彩度を示す色度を$a*,b*$で表した色空間である.また,人間の視覚に近い色空間であるため,色情報を用いて分類を行う際に有効な指標である.本研究で用いる画像は現時点ではRGB色空間にて表されているためL*a*b*色空間への変換を行う.RGB色空間からL*a*b*色空間への変換式を\Fref{Lab表色系1}と\Fref{Lab表色系2}に示す.RGB色空間はデバイス依存色であり直接L*a*b*色空間に変換する式は存在しないため,デバイス独立色であるXYZ色空間[8]に変換してから処理を行う.XYZ色空間への変換は$0$から$255$までの8bit値を$0$から$1$に正規化し,$γ=2.2$に対するガンマ補正を線形の測定値に戻す.$E$は$R,G,B$のいずれかを表し,ダッシュ付きはガンマ補正された値を示す.ただし,\Fref{Lab表色系4}を満たすとする.

      TODO: a*,b*の括弧が一つ多い
      \begin{eqnarray}
      \label{Lab表色系1}
        \left\{
          \begin{array}{l}
            L* = 116 \times f(\dfrac{Y} {Y_{n}}) - 16 \\
            a* = 500 \times [f(\dfrac{X} {X_{n}}) - f(\dfrac{Y} {Y_{n}})] \\
            b* = 200 \times [f(\dfrac{Y} {Y_{n}}) - f(\dfrac{Z} {Z_{n}})]
          \end{array}
        \right.
      \end{eqnarray}

      \begin{eqnarray}
        \label{Lab表色系2}
        \left\{
          \begin{array}{l}
            X = 0.412453R + 0.357580G + 0.180423B \\
            Y = 0.212671R + 0.715160G + 0.072169B \\
            Z = 0.019334R + 0.119193G + 0.950227B
          \end{array}
        \right.
      \end{eqnarray}

      \begin{eqnarray}
        \label{Lab表色系4}
          f(t) = 
          \left\{
            \begin{array}{lll}
              \sqrt[3]{t} 
                &(t >    (\dfrac{6} {29})^3) \\
              \dfrac{1} {3} (\dfrac{29} {6})^3 t + \dfrac{4} {29}
                &(t \leq (\dfrac{6} {29})^3)
            \end{array}
          \right.
      \end{eqnarray}

    \subsection{土砂候補領域検出}
      斜面崩壊領域は\tref{}に示すように,輝度と色相,彩度に特徴を持つ.よって,これらの特徴を表すL*a*b*色空間のL*値とa*値,HSV色空間のS値を用いる.ヒストグラム均一化処理後の画像に対しL*a*b*変換とHSV変換を行い,これらの指標を用いた閾値処理により斜面崩壊領域を検出する.

    \subsection{植生領域検出}
      \label{植生除去}
      各不要領域の特徴を\Fref{}に示す....節と同様にこれらの特徴を用いて各領域の検出を行う....項に各不要領域について示す.
      斜面崩壊は植生を多く含む山間部の斜面にて発生し,土砂に混ざった植生部分を誤検出する可能性がある.(あと普通に繁茂してるやつ)よって,\Fref{}で示した植生領域の特徴から,a*値を用いた閾値処理によって植生領域を検出する.

    \subsection{統合処理による土砂領域検出}
    \subsection{土砂領域検出結果の二値化処理}
    \subsection{土砂領域のマスク画像作成}

  
  \section{土砂量推定}
    \label{土砂量推定}
    災害後の空撮画像より作成した標高値補正済みDSMと災害前の解像度補正済み国土地理院DEMの被災前後での標高値差分を求めることにより,災害前後での土砂量変化を算出する.また,\ref{土砂マスク}節で作成した土砂マスクの領域に限定して処理を行うことにより,植生領域のずれや誤検出を除去する.
  

  \section{土砂移動推定}
    \label{土砂移動推定}
    精度評価の際にMean-Shift法での領域ベースでの正解データの目視判読による作成は領域数の多さに起因し困難であるため,メッシュベースでの土砂移動推定を行う.今回は地表面での土砂移動が確認できたランドマークの最大移動距離に合わせ,オルソ画像を縦13メッシュ,横12メッシュに区切った.

    土砂移動は一般的に上流から下流へ流下する.また,一般的に災害後には上流領域は土砂崩壊部で侵食が発生し,下流領域には上流より流下した土砂が堆積する.よって,メッシュ中心画素から以下の4条件を満たす画素を全て追跡し,メッシュ中心画素から最も距離の遠い画素へ移動線を結ぶことによって土砂移動が発生したと推定する.

    ・	上流方向から下流方向に沿って傾斜方向が向いている

    ・	上流画素の標高値が下流画素の標高値より低い
    
    ・	画素同士が隣接している
    
    ・	侵食領域と堆積領域の組み合わせである

    傾斜方向の判別には災害後DSMより作成した傾斜方向モデルを用い,侵食領域と堆積領域の組み合わせの判別には\ref{土砂量推定}節の土砂量推定結果を用いる.
  

